
\documentclass[a4paper]{article}
\usepackage[margin=1.5cm]{geometry}

\usepackage{multicol}
\usepackage{enumitem}
\usepackage{graphicx}

%Links
\usepackage[colorlinks = true,
            linkcolor = blue,
            urlcolor  = blue,
            citecolor = blue,
            anchorcolor = blue]{hyperref}

%Simbolos matemáticos
\usepackage{amsmath}
\usepackage{amssymb}

%Enumeracion
\usepackage{enumitem}

%Páginas sin numeración
\pagestyle{empty}

%Interlineado
\renewcommand{\baselinestretch}{1.5}

%Arreglar comillas
\usepackage [autostyle]{csquotes}
\MakeOuterQuote{"}

%Macros
\newcommand{\Item}{\item[\stepcounter{enumii}$\blacktriangleright$\textbf{(\alph{enumii})}]} %Negrita en algunos items
\newcommand{\answer}{\item[**]}
\newcommand{\exercise}{\item}
\newcommand{\SEL}[1]{\left\{\begin{matrix} #1 \end{matrix}\right.}
\newcommand{\df}[2]{\displaystyle\frac{#1}{#2}}

\begin{document}

\noindent \hrulefill 
\vspace{-7pt}
\begin{center} 
	\textbf{ Práctica 1: Números complejos } \\
	Comisión: Rodrigo Cossio-Pérez y Gabriel Romero
\end{center}
\vspace{-10pt}
\hrulefill


\begin{enumerate}

	\exercise Ejercicio
	%\begin{multicols}{2}
	\begin{enumerate} [label=(\alph*)]
		
		\item Inciso
		\answer Respuesta

	\end{enumerate}
	%\end{multicols}


\end{enumerate}

\end{document}