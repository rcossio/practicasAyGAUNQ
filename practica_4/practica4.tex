\documentclass{template_practica}

\begin{document}

\practiceheader{Práctica 4: Sistemas de ecuaciones}{Comisión: Rodrigo Cossio-Pérez y Gabriel Romero}

\begin{enumerate}

	\exercise Los siguientes sistemas incluyen variables reales. Simplificar los siguientes sistemas con el método de Gauss, Gauss-Jordan u otras operaciones válidas. Clasificar los sistemas con el teorema de Rouché-Frobenius. Dar la solución y el conjunto solución. 
	\begin{enumcols}[2]

		\item $\SEL{ 3x+2y=4\\ 5x-4y=6 }$
		\answer $rg(A)=rg(A^*)=n=2$. Sistema compatible determinado. \\ $(x,y)=\left(\f{14}{11},\f{1}{11}\right)$. \\ $S=\left\{(x,y)\in \R^2 ~|~ x=\f{14}{11} ~\land~ y=\f{14}{11}\right\}$. \\ Resolución por \href{https://youtu.be/OE8e70VO_CE}{Miguemáticas}

		\item $\SEL{ 2a-4b=2\\ -3a+6b=1 }$
		\answer $rg(A) = 1 \neq rg(A^*) =2$. Sistema incompatible. $S=\emptyset$. 

		\item $\SEL{ 2\alpha+7\beta+3\gamma=-7 \\ \alpha+3\beta+4\gamma=3 \\ \alpha+4\beta+3\gamma=-2 }$
		\answer $rg(A)=rg(A*)=n=3$. Sistema compatible determinado. \\ $(\alpha,\beta,\gamma)=(4,-3,2)$. \\ $S=\left\{(\alpha,\beta,\gamma)\in \R^3 ~|~ \alpha=4 ~\land~ \beta=-3 ~\land~ \gamma=2 \right\}$. \\ Resolución por \href{https://youtu.be/ZtMdsFXiFYQ}{Math2Me}

		\item $\SEL{ -2x+4y+5z=0 \\ 5x+y+3z=0 \\ 6x-y+4z=0 }$
		\answer $rg(A)=rg(A*)=n=3$. Sistema compatible determinado. \\ $(x,y,z)=(0,0,0)$. \\ $S=\left\{(x,y,z)\in \R^3 ~|~ x=y=z=0 \right\}$. \\ Resolución por \href{https://youtu.be/5tOyCI7YIwk?t=21}{Cátedra de Matemática}

		\item $\SEL{ k_1+k_2+k_3=2 \\ 5k_1+k_2-4k_3=1 \\ -7k_1+2k_2-5k_3=3 }$
		\answer $rg(A)=rg(A*)=n=3$. Sistema compatible determinado. \\ $(k_1,k_2,k_3)=\left(-\f{2}{73},\f{135}{73},\f{13}{73}\right)$. \\ $S=\left\{(k_1,k_2,k_3)\in \R^3 ~|~ k_1=-\f{2}{73} ~\land~ k_2=\f{135}{73} ~\land~ k_3=\f{13}{73} \right\}$. \\ Resolución por \href{https://youtu.be/SxT5Sbn8odE}{JulioProfe}

		\item $\SEL{ 2P+Q-R=1 \\ 3P-2Q+R=2 \\ -4P+3Q-2R=-4 }$
		\answer  $rg(A)=rg(A*)=n=3$. Sistema compatible determinado. \\ $(P,Q,R)=(1,2,3)$. \\ $S=\left\{(P,Q,R)\in \R^3 ~|~ P=1 ~\land~ Q=2 ~\land~ R=3 \right\}$. \\ Resolución por \href{https://youtu.be/kpRQ_jWHSqg?t=95}{Miguemáticas}

		\item $\SEL{ x-y+3z=4 \\ 2x-y-z=6 \\ 3x-2y+2z=10 }$
		\answer  $rg(A)=rg(A)=2<n=3$. Sistema compatible indeterminado cuya solución depende de 1 parámetro. \\ $(x,y,z)=(2+4z,-2+7z,z)$. \\ $S=\left\{(x,y,z)\in \R^3 ~|~ x=2+4z ~\land~ y=-2+7z \right\}$. \\ Resolución por \href{https://youtu.be/ERUAPI-jrH0}{Mates con Andrés}

		\item $\SEL{ 2a+2b+3c=0 \\ 2a+5b+c=0 }$
		\answer $rg(A)=rg(A)=2<n=3$. Sistema compatible indeterminado cuya solución depende de 1 parámetro. $(a,b,c)=\left(-\f{13}{6}c,\f{2}{3}c,c\right)$. \\ $S=\left\{(a,b,c)\in \R^3 ~|~ a=-\f{13}{6}c ~\land~ b=\f{2}{3}c \right\}$. \\ Resolución por \href{https://youtu.be/etAvBSvqpB0}{Cátedra de Matemática}

		\item $\SEL{ 3x+2y+z-w=3 \\ x+y-2z-3w=5 \\ 7x+8y +7z +3w=2 }$
		\answer $rg(A)=rg(A)=3<n=4$. Sistema compatible indeterminado cuya solución depende de 1 parámetro. $(x,y,z,w)=\left( \f{29}{28}+\f{5}{7}w,\f{3}{4},-\f{45}{28}-\f{8}{7}w,w \right)$. \\ $S=\left\{(x,y,z,w) \in \R^4 ~|~ x=\f{29}{28}+\f{5}{7}w ~\land~ y=\f{3}{4} ~\land~ z=-\f{45}{28}-\f{8}{7}w \right\}$. \\ Resolución por \href{https://youtu.be/OE8e70VO_CE}{Ktipio}

		\item $\SEL{ 3T+P=2 \\ 6T-P=1 \\ 2T+2P=3 }$
		\answer $rg(A)=2 \neq rg(A^*)=3$. Sistema incompatible. $S=\emptyset$. \\ Resolución por \href{https://youtu.be/u-F-3jxqf_s}{Pa q aprendas}

		\item $\SEL{ x+2y=6 \\ 2x+y=6 \\ 3x+4y=14 }$
		\answer $rg(A)=rg(A*)=n=2$. Sistema compatible determinado. \\ $(x,y)=(2,2)$. $S=\left\{(x,y)\in \R^2 ~|~ x=y=2 \right\}$. \\ Resolución por \href{https://youtu.be/qTj1yLgl1F4?t=131}{Mate Profesor Rosado}

		\item $\SEL{ 2x-y+15z=3 \\ x-3y-2z=7 \\ x-8y-21z=11 }$
		\answer $rg(A)=2 \neq rg(A^*)=3$. Sistema incompatible. $S=\emptyset$. \\ Resolución por \href{https://youtu.be/__8Vg4ePRr4}{Mates con Andrés}

		\item $\SEL{ 2x-2y+14z-2w=0 \\ 4x-6y-16z+2w=0 }$
		\answer $rg(A)=rg(A)=2<n=4$. Sistema compatible indeterminado cuya solución depende de 2 parámetros. \\ $(x,y,z,w)=\left( -29z+4w,-22z+3w,z,w \right)$. $S=\left\{(x,y,z,w) \in \R^4 ~|~ x=-29z+4w ~\land~ y=-22z+3w \right\}$. \\ Resolución por \href{https://youtu.be/XqKrK9zkwtE}{Ktipio}

	\end{enumcols}



	\exercise Las siguientes matrices ampliadas son el resultado de aplicar el método de Gauss-Jordan. Clasificar los sistemas con el teorema de Rouché-Frobenius. Dar la solución y el conjunto solución. En los casos donde el sistema sea compatible indeterminado dar 3 ejemplo de soluciones particulares distintas.
	\begin{enumcols}[2]

		\item $\AMat{2}{ 1 & 0 & -5 \\ 0 & 1 & 2 }$
		\answer $rg(A)=rg(A^*)=n=2$. Sistema compatible determinado. \\ $(x,y)=(-5,2)$. \\ $S=\left\{(x,y)\in \R^2 ~|~ x=-5 ~\land~ y=2 \right\}$. 

		\item $\AMat{2}{ 1 & 0 & 0 \\ 0 & 0 & 3 }$
		\answer $rg(A)=1 \neq rg(A*)=2$. Sistema incompatible. $S=\emptyset$.

		\item $\AMat{2}{ 1 & 0 & 4 \\ 0 & 0 & 0 }$
		\answer $rg(A)=rg(A^*)=1 < n=2$. Sistema compatible indeterminado cuya solución depende de 1 parámetro. \\ $(x,y)=(4,y)$. \\ $S=\left\{(x,y)\in \R^2 ~|~ x=4 \right\}$. \\ Con $y=1 \to (4,1)$. Con $y=4 \to (4,4)$. Con $y=-5 \to (4,-5)$. 

		\item $\AMat{3}{ 1 & -2 & 0 & 7 \\ 0 & 3 & 1 & -5 \\ 0 & 0 & 0 & 0 }$
		\answer $rg(A)=rg(A^*)=2<n=3$. Sistema compatible indeterminado cuya solución depende de 1 parámetro. \\ $(x,y,z)=(7+2y,y,-5-3y)$. \\ $S=\left\{(x,y,z)\in \R^3 ~|~ x=7+2y ~\land~ z=-5-3y \right\}$. \\ Con $y=0 \to (7,0,-5)$. Con $y=2 \to (11,2,-11)$. Con $y=1 \to (9,1,-8)$.   
		
		\item $\AMat{3}{ 1 & 0 & -1 & 7 \\ 0 & 1 & 2 & 0 \\ 0 & 0 & 0 & 0 \\ 0 & 0 & 0 & 0 }$
		\answer $rg(A)=rg(A^*)=2<n=3$. Sistema compatible indeterminado cuya solución depende de 1 parámetro. \\ $(x,y,z)=(7+z,-2z,z)$. \\ $S=\left\{(x,y,z)\in \R^3 ~|~ x=7+z ~\land~ y=-2z \right\}$. \\ Con $z=0 \to (7,0,0)$. Con $z=1 \to (8,-2,1)$. Con $z=2 \to (9,-4,2)$. 

		\item $\AMat{4}{ 1 & -2 & 0 & 0 & -6 \\ 0 & 0 & 1 & 0 & 2 \\ 0 & 0 & 0 & 1 & 0 }$
		\answer $rg(A)=rg(A^*)=3<n=4$. Sistema compatible indeterminado cuya solución depende de 1 parámetro. \\ $(x,y,z,w)=(-6+2y,y,2,0)$. \\ $S=\left\{(x,y,z,w)\in \R^4 ~|~ x=-6+2y ~\land~ z=2 ~\land~ w=0 \right\}$. \\ Con $y=5 \to (4,5,2,0)$. Con $y=4 \to (2,4,2,0)$. Con $y=3 \to (0,3,2,0)$.

		\item $\AMat{4}{ 1 & -1 & 2 & 3 & -2 \\ 0 & 0 & 0 & 0 & 3 \\ 0 & 0 & 0 & 0 & 0 }$
		\answer $rg(A)=1 \neq rg(A^*)=2$. Sistema incompatible. $S=\emptyset$.

		\item $\AMat{4}{ 1 & 0 & 0 & -1 & \f{1}{2} \\ 0 & 0 & 1 & \f{3}{4} & 2 \\ 0 & 0 & 0 & 0 & 0 }$
		\answer $rg(A)=rg(A^*)=2<n=4$. Sistema compatible indeterminado cuya solución depende de 2 parámetros. \\ $(x,y,z,w)=(\f{1}{2}+w,y,2-\f{3}{4}w,w)$. \\ $S=\left\{(x,y,z,w)\in \R^4 ~|~ x=\f{1}{2}+w ~\land~ z=\f{3}{4}w ~\land~ w=0 \right\}$. \\ Con $y=1$ y $w=1$ $\to (\f{3}{2},1,\f{5}{4},1)$. Con $y=1$ y $w=2$ $\to (\f{5}{2},1,\f{1}{2},2)$. Con $y=2$ y $w=0$ $\to (\f{1}{2},2,2,0)$.

	\end{enumcols}



	\exercise Resolver los siguientes acetijos
	\begin{enumcols}[2]
		
			\item \img{1.05\textwidth}{img/table_puzzle.png}
			\item \img{\textwidth}{img/fruit_puzzle.png}
			\item \img{0.6\textwidth}{img/square_puzzle.png}
			\item \img{\textwidth}{img/fruit_puzzle2.png}
			\item \img{\textwidth}{img/weight_puzzle.png}
		
		\end{enumcols}



	\exercise Plantear ecuaciones o sistemas de ecuaciones que modelen las situaciones. Resolverlas/los y responder a lo pedido. \textit{Nota: Los problemas usan datos reales o verosímiles pero tienen cierto grado de simplificación para no complicar los problemas}
	\begin{enumcols}

		\item En promedio, por año, una hectárea de manzanos produce 24,6 toneladas de manzana mientras que una hectárea de perales produce 27,1 toneladas de pera. ¿Cuántas toneladas de fruta se obtienen si se siembran 3 hectareas de manzanos y 5 hectareas de perales? ¿Cuántas hectáreas hay que plantar de cada árbol para obtener 400 toneladas de fruta al año?
		%\textit{Fuente: Cadena de valor Manzana y Pera. Ministerio de Hacienda. Secretaría de Política Económica. Diciembre 2017. ISSN 2525-0221.}
		\answer Si llamamos $m$ a las hectáreas de manzana, $p$ a las hectáreas de peras y $f$ a las toneladas de fruta obtenidas obtenemos la ecuación $24,6 m + 27,1 p = f$. Con $m=3$ y $p=5$, tendremos $f=24,6 (3)+27,1(5)=209,3$ toneladas de fruta. Con $f=400$ tenemos la ecuación $24,6 m + 27,1 p = 400$, de donde se puede despejar $p=-\f{246}{271}m+\f{4000}{271}$. La solución es $(m,p)=\left(m,-\f{246}{271}m+\f{4000}{271}\right)$, es decir, para obtener 400 toneladas de fruta hay infinitas soluciones, la cantidad de perales dependerá de la de manzanos. Por ejemplo, si se plantan 10 hectáreas de manzanos, se necesitarán $\sim5,683$ hectáreas de perales.

		\item Un puesto de pochoclos vende bolsitas pequeñas de pochoclos a \$800 cada una y bolsas grandes de pochoclos a \$1000 cada una. En un día particular, el puesto vendió un total de 120 bolsas de pochoclos por un total de \$102.000. ¿Cuántas bolsitas pequeñas y cuántas bolsas grandes vendió el puesto ese día?
		\answer Llamamos $p$ a la cantidad de bolsitas pequeñas y $g$ a la cantidad de bolsitas grandes. Tenemos el sistema de ecuaciones \\ \vspace{2mm} $\SEL{ p+g=120 \\ 800p+1000g=102000 }$ \\ Resolviendo el sistema obtenemos que se vendieron 90 bolsitas pequeñas y 30 bolsas grandes.

		\item En una empresa de transporte, se tienen camiones de larga distancia que pueden llevar 33 toneladas y produce 0,057 kg de CO$_2$ por tonelada de carga y por kilómetro. También se cuenta con camiones de reparto urbano que pueden llevar 18 toneladas y producen 0,307 kg de CO$_2$ por tonelada de carga y por kilómetro. Se desean transportar cargas a la ciudad de Mendoza, que está a 1050~km de Buenos Aires. ¿Cuántas toneladas se pueden llevar y cuánto CO$_2$ se produce si se realizan 10 viajes de camiones de larga distancia y 8 urbanos? Por reportes de la empresa, se conoce que se llevaron 678~toneladas de carga a la ciudad de Mendoza y se produjeron 97278,3~kg de CO$_2$. ¿Cuántos viajes de de cada tipo de camión se realizaron? 
		%\textit{Fuente: \href{https://theicct.org/publication/co2-emissions-from-trucks-in-the-eu-an-analysis-of-the-heavy-duty-co2-standards-baseline-data/}{ICCT}}
		\answer Llamamos $x$ a la cantidad de viajes de camiones de larga distancia e $y$ a la cantidad de viajes de camiones urbanos. Considerando un viaje de 1050~km y las toneladas que puede llevar cada camión sabemos que se producen $0,057~.~1050~.~33=1975,05$ kg de CO$_2$ por viaje en camión de larga distancia y $0,307~.~1050~.~18=5802,3$ kg de CO$_2$ por viaje en camiones urbanos. Podemos armar una ecuación para considerar las toneladas de carga ($t$)y una para considerar el CO$_2$ ($c$). Obtenemos el siguiente sistema de ecuaciones: \\ \vspace{2mm} $\SEL{ 33x+18y=t \\ 1975,05x+5802,3y=c }$ \\ Por lo que, con $x=10$ e $y=8$ tenemos 474 toneladas y 66168,9~kg de CO$_2$. Por otra parte, si consideramos $t=678$ y $c=97278,3$ tenemos el sistema \\ \vspace{2mm} $\SEL{ 33x+18y=678 \\ 1975,05x+5802,3y=97278,3 }$ \\ La solución es $(x,y)=(14,12)$, es decir, se realizaron 14 viajes de camiones de larga distancia y 12 viajes de camiones urbanos.

		\item Un sistema de viandas ofrece 3 tipos de viandas: vianda vegetariana a \$900, vianda con carne a \$1200 y vianda con pollo a \$1100.  Las viandas se empacan en un contenedor de 42~L que soporta hasta 20~kg. El empaque de las viandas de carne y pollo es de 0,8~L mientras que el de la vianda vegetariana es de 1,1~L. Las viandas de pollo y vegetariana pesan 400g mientras que la de carne pesa 600g. Una empresa solicitó 45 viandas para sus empleados/as que ocuparon la capacidad máxima (en volumen y peso) del contenedor del envío. ¿Cuántas viandas de cada tipo se pidieron y cuanto costó el total?
		\answer Llamamos $x$, $y$ y $z$ a las viandas vegetarianas, con carne y con pollo, respectivamente. Consideramos la cantidad de viandas, su volumen y peso (de momento, ignoramos los precios). Se obtiene el sistema: \\ \vspace{2mm} $\SEL{ x+y+z=45 \\ x+0,8y+0,8z=42 \\ 0,4x+0,6y+0,4z=20 }$ \\ La solución es $(x,y,z)=(30,10,5)$, es decir, se pidieron 30 viandas vegetarianas, 10 con carne y 5 con pollo. Con las viandas calculamos el precio total: $900x+1200y+1100z=900(30)+1200(10)+1100(5)=44500$ pesos.

		\item La combustión del metano (CH$_4$) consume oxígeno (O$_2$) y produce dióxido de carbono (CO$_2$) y agua (H$_2$O) según la reacción: \\ \phantom{-------------------------------------------------}$ aCH_4 + bO_2 \to cCO_2 + d H_{2}O$ \\ donde $a$, $b$, $c$ y $d$ de denominan coeficientes estequiométricos. Si se queman $a$~moles de metano se consumen $b$ moles de oxigeno obtienen $c$ moles de CO$_2$ y $d$ moles de agua. Considerando la ley de conservación de masa, hallar coeficientes válidos para la reacción. ¿Cuántos moles de CO$_2$ y de H$_2$O se producen si se queman 5~moles de metano? 
		\answer En la reacción tenemos tres átomos que se conservan (C,O,H), por que lo obtenemos una ecuación para cada uno: \\ \vspace{2mm} $\SEL { a-c = 0 \\ 2b -2c -d = 0 \\ 4a - 2d = 0 }$ \\ La solución del sistema es $(a,b,c,d) = (a,2a,a,2a)$, por lo que elegimos $a=1$, y la reacción queda $ CH_4 + 2O_2 \to CO_2 + 2 H_{2}O$. Si se consumen 5 moles de metano, $a=5$ y la solución será $(5,10,5,10)$, es decir, se formaran 5 moles de CO$_2$ y 10 moles de H$_2$O.

	\end{enumcols}


\end{enumerate}

\end{document}