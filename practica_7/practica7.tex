\documentclass{template_practica}

\begin{document}

\practiceheader{Práctica 7: Rectas y Planos}{Comisión: Rodrigo Cossio-Pérez y Gabriel Romero}

\begin{enumerate}

	\exercise Hallar una recta del plano ($\R^2$) que cumpla con las siguientes condiciones. Dar una ecuación implícita o general, explícita, paramétrica vectorial, paramétrica cartesiana y simétrica. Indicar si la recta hallada es única.
	\begin{enumcols}
		
		\item Pasa por el punto $P(-3,1)$ y es paralela al vector $\vec{d}=(2,-1)$
		\answer La recta es única. \\ Algunas ecuaciones: $(x,y)=(-3,1)+k(2,-1)$. \\ $\SEL{ x=-3+2k \\ y=1-k \hfill }$ \\ $\f{x+3}{2}=\f{y-1}{-1}$.

		\item Pasa por el punto $P(1,2)$ y es paralela al vector $\vect{AB}$ con $A(2,-5)$ y $B(4,-5)$
		\answer La recta es única. Se utiliza $\vect{AB}=(2,0)$ como director. \\ $(x,y)=(1,2)+k(2,0)$. \\ $\SEL{ x=1+2k \\ y=2 \hfill }$ \\ $y=2$.\\ $y-2=0$. \\ No existe simétrica.

		\item Es coincidente con la recta $L: 2x-y+8=0$
		\answer La recta es única. $4x-2y+16=0$. $y=2x+8$. $(x,y)=(0,8)+k(1,2)$. \\ $\SEL{ x=k \hfill \\ y=8+2k }$ \\ $x=\f{y-8}{2}$.

		\item Pasa por los puntos $P(-3,0)$ y $Q(-3,3)$
		\answer La recta es única. Se utiliza $\vect{PQ}=(0,3)$ como director. \\ $(x,y)=(-3,0)+k(0,3)$  \\ $\SEL{ x=-3 \\ y=3k }$ \\ $x=-3$. \\ $x+3=0$. \\ No existe simétrica.

		\item Pasa por el punto $P(1,4)$ y pendiente $-\f{1}{2}$
		\answer La recta es única. $y=-\f{1}{2}x+\f{9}{2}$. $(x,y)=(1,4)+k(2,-1)$. \\ $\SEL{ x=1+2k \\ y=4-k }$ \\ $\f{x-1}{2}=\f{y-4}{-1}$.

		\item Pasa por el origen y es perpendicular al vector $\vec{n}=(2,3)$
		\answer La recta es única. Se utiliza $\vec{d}=(-3,2)$ como director ya que $\vec{n} \perp \vec{d}$. \\ $(x,y)=k(-3,2)$. \\ $\SEL{ x=-3k \\ y=2k }$ \\ $\f{x}{-3}=\f{y}{2}$.

		\item Pasa por el punto $P(1,2)$ y es perpendicular al eje de abscisas (eje $x$)
		\answer La recta es única. Se utiliza $\jvec$ como director ya que $\jvec \perp \ivec$. \\ $(x,y)=(1,2)+k(0,1)$. \\ $\SEL{ x=1 \hfill\\ y=2+k }$ \\ $x=1$. \\ $x-1=0$. \\ No existe simétrica.

		\item Tiene a -4 como absisa al origen y 3 como ordenada al origen
		\answer La recta es única. De los puntos $A(-4,0)$ y $B(0,3)$ se obtiene el vector director $\vect{AB}=(4,3)$. \\ $(x,y)=(0,3)+k(4,3)$. \\ $\SEL{ x=4k \hfill \\ y=3+3k }$ \\ $\f{x}{4}=\f{y-3}{3}$.

		\item Pasa por los puntos $P(-3,1)$, $Q(1,3)$, $R(3,-3)$
		\answer No existe una recta que cumpla con las condiciones ya que los puntos no están alineados. Se puede ver porque $\vect{PQ} \nparallel\vect{QR}$

		\item Corta al eje de ordenadas en el punto $M(0,2)$
		\answer La recta NO es única, la misma responde a la ecuación $y=mx+2$ o $(x,y)=(0,2)+k(a,b)$. Por ejemplo, se elije $m=1$ y se obtiene la recta $y=x+2$.

		\item Forma un ángulo de $30\degs$ con el eje de abscisas (eje $x$).
		\answer La recta NO es única. Se obtiene el vector unitario $\hat{d}=(\cos 30\degs, \sin 30\degs)=\left(\f{\sqrt{3}}{2},\f{1}{2}\right)$ como director. Se puede elegir cualquier punto para definir la recta. Por ejemplo, con $P(1,3)$ y se obtiene la recta $(x,y)=(1,3)+k\left(\f{\sqrt{3}}{2},\f{1}{2}\right)$.

		\item Pasa por el punto $P(3,-2)$ y es normal a $\vec{n}$, donde $\vec{n}$ es un vector con ángulo $\f{\pi}{4}$ con respecto al eje de ordenadas.
		\answer $\vec{n}$ puede ser $\vect{n_1}=(\cos 45\degs, \sin 45\degs)=\left(\f{\sqrt{2}}{2},\f{\sqrt{2}}{2}\right)$ o $\vect{n_2}=(\cos 135\degs, \sin 135\degs)=\left(-\f{\sqrt{2}}{2},\f{\sqrt{2}}{2}\right)$. Por lo que la recta no es única. Las opciones son: $(x,y)=(3,-2)+k\left(\f{\sqrt{2}}{2},\f{\sqrt{2}}{2}\right)$ o $(x,y)=(3,-2)+t\left(-\f{\sqrt{2}}{2},\f{\sqrt{2}}{2}\right)$.

	\end{enumcols}


	\exercise Analizar las posiciones relativas de las rectas del plano. Es decir, analizar si son paralelas, coincidentes, perpendiculares o incidentes. Calcular la intersección entre las rectas.
	\begin{enumcols}
		
		\item $s: 2x+y-1=0$ ~~y~~ $r:\SEL{x=1+2t \\ y=3-t}$
		\answer $s \parallel r$. $s \cap r=\emptyset$.  Resolución por \href{https://youtu.be/cHsXMw3V4mQ}{Elena Virseda}

		\item $r:\SEL{x=-3+2t \\ 4-t}$ ~~y~~ $s:\f{x+6}{-4}=\f{y-1}{2}$.
		\answer $r \parallel s$. $r \cap s=\emptyset$. Resolución por \href{https://youtu.be/MPUh70MKVUY}{Matemáticas y Física con Pablo}

		\item $r_1: 2x-4y=2$ ~~y~~ $r_2: x+y=0$.
		\answer $r_1$ y $r_2$ son incidentes no-perpendiculares ya que se intersectan en un punto y sus normales no son perpendiculares: $(2,-4) \not\perp (1,1)$. La intersección es $r_1 \cap r_2=\left\{ \left(\f{1}{3},-\f{1}{3}\right)\right\}$. \href{https://youtu.be/LgIU8pd8DeM?t=97}{Matemáticas con Huais}

		\item $r_1: x+3y=7$ ~~y~~ $r_2: (x,y)=(4,1)+\alpha (-6,2)$ con $\alpha \in\R$.
		\answer $r_1$ y $r_2$ son coincidentes. La intersección es $r_1 \cap r_2=r_1=r_2$. \href{https://youtu.be/LgIU8pd8DeM?t=884}{Matemáticas con Huais}

		\item $r: 2x-y-3=0$ ~~y~~ $s: (x,y)=(-1,0)+ k (-6,3)$ con $k\in\R$.
		\answer Reemplazamos $\SEL{x=-1-6k \\ y=3k \hfill}$ en $2x-y-3=0$ y obtenemos $k=-\f{1}{3}$. Por lo tanto, $r \cap s=\left\{ \left(1,-1\right)\right\}$. Además $\vect{n_r}=(2,-1)$ es paralela a $\vect{d_s}=(-6,3)$, por lo que $r \perp s$.

		\item $r: y=-\f{1}{2}x+\f{3}{2}$ ~~y~~ $s: \f{1}{4}x+\f{3}{4}y-3=0$.
		\answer Considerando que $\vect{n_r}=(1,2)$ y $\vect{n_s}=(1,3)$ se puede ver que $\vect{n_r}\not\perp\vect{n_s}$ y $\vect{n_r}\nparallel\vect{n_s}$, por lo que las rectas son incidentes con intersección $r\cap s=\{(-15,9)\}$ y ángulo entre rectas $\sim 8.13\degs$. 

		\item $r: (x,y)=(4,2-3k)$ ~~y~~ $s: (x,y)=(2-4h,-3)$ con $k,h\in\R$.
		\answer Considerando las rectas $r: (x,y)=(4,2) +k(0,-3)$ ~~y~~ $s: (x,y)=(2,-3) +h(-4h,0)$ se puede observar que $\vect{v_r} \perp \vect{v_s}$, por lo que las rectas son perpendiculares con intersección $r\cap s=\{(4,-3)\}$.

		\item $r: (x,y)=k(3,-2)+(10,5)$ con $k\in\R$ ~~y~~ $s: 2x+3y-35=0$.
		\answer Se puede observar que $\vect{v_r} \perp \vect{n_s}$ y que $(10,5) \in s$, por lo que las rectas son coincidentes.

	\end{enumcols}

	
	\exercise Graficar los siguientes lugares geométricos relacionados a rectas.
	\begin{enumcols}[2]

		\item $4x+3y > 4$
		\answer \img{55mm}{plots/plot2.png}

		\item $4x+3y \neq 4$
		\answer \img{55mm}{plots/plot3.png}

		\item $2x-y \leq 6$
		\answer \img{55mm}{plots/plot4.png}

		\item $2x-y < 6$
		\answer \img{55mm}{plots/plot5.png}

		\item $(x,y)=k(1,2)+(-2,3)$ con $k\in[0,+\infty)$
		\answer \img{55mm}{plots/plot1.png}

		\item $(x,y)=\lambda(1,2)+(0,3)$ con $\lambda\in\mathbb{Z}$
		\answer \img{55mm}{plots/plot6.png}

		\item $(x,y)=k(1,2)+(-1,3)$ con $k\in[-1,1]$
		\answer \img{55mm}{plots/plot7.png}

		\item $y=mx+2$ con $m\in[0,1)$
		\answer \img{55mm}{plots/plot8.png}

		\item $y=2x+b$ con $b\in[-2,1]$
		\answer \img{55mm}{plots/plot9.png}

		\item $R=\{(x,y) \in \R^2 ~|~ -x+y-5=0 ~\land~ 2x+y-17=0 \}$
		\answer \img{55mm}{plots/plot10.png}

		\item $S=\{(x,y) \in \R^2 ~|~ -x+y-5=0 ~\lor~ 2x+y-17=0 \}$
		\answer \img{55mm}{plots/plot11.png}

		\item $\SEL{-x+y-5=0 \\ 2x+y-17=0 \\ x+2y-10=0 }$

		\item $\SEL{-x+y-5=0 \\ 2x+y-17=0 \\ x+2y-17=0 \\ 3x+3y-34=0 }$
	
	\end{enumcols}


	\exercise Resolver los siguientes problemas integradores relacionados a la geometría en el plano.
	\begin{enumcols}
		
		\item Calcular la distancia entre el origen de coordenadas y la recta $r: 3x+4y -24=0$.
		\answer $d(O,r)=\f{|3.0+4.0-24|}{\sqrt{3^2+4^2}}=\f{24}{5}$.

		\item Calcular la distancia entre el punto $(-1,7)$ y la recta $r: x-3y -10=0$.
		\answer $d(P,r)=\f{|1(-1)-3(7)-10|}{\sqrt{(-1)^2+7^2}}=\f{32}{\sqrt{50}}$.

		\item Encontrar un punto que equidiste de las rectas $r: x+2y-3=0$ y $s: -x-2y+4=0$ y otro que equidiste de las rectas $r$ y $t: 2x-y+1=0$.
		\answer Si planteamos $d(P,r)=d(P,s)$ se obtiene $\f{|x+2y-3|}{\sqrt{1^2+2^2}}=\f{|-x-2y+4|}{\sqrt{(-1)^2+(-2)^2}}$, de lo que se despeja $(x+2y-3)=\pm(-x-2y+4)$. Una de las alternativas es absurda y la otra es la recta $2x+4y-7=0$, por lo que podemos elegir el punto $P\left(\frac{3}{2},1\right)$. Como $r$ y $s$ son paralelas, gráficamente los puntos buscados están a "la mitad" entre ellas. Con respecto a un punto que equidiste de las rectas $r$ y $t$ de la misma manera obtenemos las rectas $3x+y-2=0$ y $-x+3y-4=0$, por lo que elegimos un punto de alguna: $Q(1,-1)$. Como las rectas $r$ y $t$ son incidentes los puntos buscados son las bisectrices de ellas.

		\item Calcular la distancia entre la recta que pasa por (1,-4) de pendiente -2 y la recta $r: 2x+y-6=0$.
		\answer Orientación: Primero verificar que las rectas son paralelas para asegurar que efectivamente hay una distancia entre ellas. Luego la distancia entre las rectas $r_1$ y $r_2$ será la misma que entre un punto de $r_1$ y $r_2$.

		\item Calcular la distancia entre las rectas $r: (x,y)=t(4,4)+(2,-5)$ y $s: (x,y)=k(1,1)+(0,-9)$

		\item Calcular la proyección ortogonal del punto $P(3,-1)$ sobre la recta $r: -x-y-12=0$.
		\answer Guía: La proyección ortogonal de un punto $P$ sobre una recta es el punto de la recta más cercano al punto. Para encontrarlo se puede puede plantear una recta auxiliar perpendicular a la recta que pase por el punto $P$ y el punto proyectado estará en la intersección entre las rectas (hacer un grafico para comprender esta situación)

		\item Calcular la proyección ortogonal del punto $P(-2,2)$ sobre la recta $r: -5x+y-1=0$ y el punto simétrico $P'$.
		\answer Orientación: Una vez hallada la proyección ortogonal $Q$ se puede hallar $P'$ considerando que $\vect{PQ}=\vect{QP'}$

		\item Hallar el valor de $a\in\R$ para que el ángulo entre $r_1: ax+3y=0$ ~y~ $r_2:\SEL{ x=4+\lambda \\ 1-2\lambda }$ sea de $30\degs$. 
		\answer Guía: El ángulo entre dos rectas es el mismo que entre sus vectores directores. Por lo que se puede plantear dicha ecuación utilizando un $a$ genérico y un ángulo de $30\degs$, y luego calcular los valores de $a$.

	\end{enumcols}


	\exercise Hallar una recta del espacio ($\R^3$) que cumpla con las siguientes condiciones. Dar una ecuación paramétrica vectorial, paramétrica cartesiana y simétrica. Indicar si la recta hallada es única.
	\begin{enumcols}

		\item Pasa por los puntos $A(2,-3,1)$ y $B(-3,5,0)$.

		\item Es paralela al vector $(2,5,-2)$ y contiene al punto $P(4,-5,0)$.

		\item Incluye al punto $P(4,-5,0)$ y es perpendicular al vector $(2,5,-2)$.

		\item Es perpendicular a los vectores $(-2,0,0)$ y $(0,0,-6)$ y pasa por el punto $P(3,3,0)$

		\item Es paralela a $r: (x,y,z)=(2,4,-1)+t(-6,0,2)$ y pasa por $P(4,1,5)$.

		\item Es perpendicular a $r: (x,y,z)=(2,4,-1)+t(-6,0,2)$ y contiene a $P(4,1,5)$.

		\item Pasa por el punto $P(3,3,0)$ y es paralela a la recta $r: \SEL{y=z\\x=0}$

		\item Es perpendicular a $\pi: x+y+z-3=0$

		\item Es perpendicular a $\pi: 3x-2y+z=5$ y incluye a $P(4,1,5)$

		\item Es paralela a $\pi: 3x-2y+z=5$ y pasa por $P(4,1,5)$

		\item Es perpendicular a $r_1: \f{x+2}{2}=-y+3=\f{z+2}{5}$ y a $r_2: x-3=\f{2y-7}{2}=\f{z-3}{3}$ y contiene al punto $P(3,-3,4)$
		\answer $r: (x,y,z)=(3,-3,4)+t(-8,-1,3)$. \\ $r: \SEL{x=3-8y\\y=-3-t\\z=4+3t}$ \\ $r: \f{x-3}{-8}=\f{y+3}{-1}=\f{z-4}{3}$. \\ La recta es única. Resolución por \href{https://youtu.be/KebOzsUUmq4?t=458}{Mate316}.

		\item Pasa por el origen de coordenadas, es perpendicular a $r: (x,y,z)=(1,4,1)+t(3,-5,0)$ y no incluye a $P(0,0,3)$

		\item Está contenida en el plano $\pi: 2x-3y+2z=6$ y es perpendicular a la recta $r: (x,y,z)=(2,-9,7)+t(1,-1,1)$.

		\item Es paralela a los planos $\pi_1: 2x-3y+5z=2$ y $\pi_2: -x+3y+2z=1$ e incluye al punto $P(2,1,0)$.

	\end{enumcols}


	\exercise Hallar un plano que cumpla con las siguientes condiciones. Dar una ecuación implícita o general, explícita, paramétrica vectorial y paramétrica cartesiana. Indicar si el plano hallado es única.
	\begin{enumcols}

		\item  Pasa por el punto $P(2,2,2)$ y es perpendicular al vector $(3,-1,-2)$.
		\answer Orientación: Hay una ecuación general del plano $Ax+By+Cz+D=0$ contiene información directa sobre el vector normal $\vec{n}=(A,B,C)$. Sólo resta hallar $D$, utilizando el punto $P$. 

		\item  Es paralelo a los vectores $(2,0,2)$ y $(0,3,0)$ y pasa por el punto $P(1,2,0)$.
		\answer Pista: La ecuación paramétrica vectorial de un plano requiere de dos vectores paralelos al plano y un punto del mismo.

		\item  Pasa por los puntos $A(-1,1,1)$, $B(1,0,2)$ y $C(0,2,-2)$
		\answer Orientación: La ecuación paramétrica vectorial de un plano utiliza dos vectores paralelos la plano (pero no paralelos entre si) y un punto del mismo. Se pueden crear dos vectores a partir de tres puntos, como $\vect{AB}$, $\vect{BC}$, $\vect{AC}$, etc. 

		\item  Es perpendicular al vector $(-2,0,1)$.
		\answer Guía: Plantear una ecuación del plano donde se pueda utilizar el vector normal. Pensar geométricamente el problema para entender qué condiciones cumple el plano hallado y si es único. En caso de que no sea unico, elegir arbitrariamente los datos que faltan para dar la ecuación de un plano (sin dejar de cumplir las condiciones previas pedidas)

		\item  Es paralelo al vector $(2,-1,4)$ y pasa por el punto $P(3,3,1)$.
		\answer Orientación: La ecuación paramétrica vectorial se plantea con un punto del plano y dos vectores paralelos al mismo. Con los datos dados, se pueden reemplazar los datos existentes y pensar si los faltantes pueden calcularse o deben ser elegidos arbitrariamente.

		\item  Pasa por los puntos $P(4,0,5)$ y $Q(-1,-1,0)$.
		\answer Guía: Dado que $P$ y $Q$ pertenecen al plano el vector $\vect{PQ}$ derá paralelo al plano. Plantear la ecuación paramétrica vectorial con los datos conocidos y pensar si los faltantes pueden calcularse o deben ser elegidos arbitrariamente.

		\item  Pasa por los puntos $A(0,1,1)$, $B(1,3,2)$, $C(0,2,-4)$ y $D(1,7,5)$.
		\answer Orientación: Utilizando los puntos $A$, $B$ y $C$ se puede plantear la ecuación paramétrica vectorial del plano. Luego, se puede verificar si el punto $D$ pertenece al plano hallado. Alternativa: reemplazar los puntos en la ecuación general $ax+by+cz+d=0$ y hallar los valores de $a$, $b$, $c$ y $d$ del sistema de cuatro ecuaciones formado.

		\item  Pasa por los puntos $P(3,-2,0)$ y $Q(1,-1,1)$ y es paralelo al vector $(1,2,2)$.
		\answer Guía: La ecuación paramétrica vectorial utiliza dos vectores del plano que no sean paralelos entre si y un punto. Utilizar alguno de los puntos dados y el vector $(1,2,2)$ en la ecuación. Luego, verificar si $\vect{PQ}$ es paralelo a $(1,2,2)$. En caso de no ser paralelo, se puede utilizar en la ecuación.

		\item  Es paralelo al plano $\pi: 2x-3y+2z=6$ y pasa por el punto $P(2,1,0)$.
		\answer Orientación: Verificar qué relación tienen las normales de planos paralelos y analizar si la información, junto con $P$, es suficiente para definir un plano único.

		\item  Es paralelo a las rectas $r_1: (x,y,z)=(1,4,0)+t(5,5,-3)$ y $r_2: (x,y,z)=(-10,2,7)+t(4,0,3)$ y contiene al punto $P(2,2,2)$.
		\answer Guía: Para definir un plano necesitamos un punto y una normal (ec. general) o un punto y dos vectores paralelos (ec. paramétrica vectorial). Analizar qué información de puede sacar del plano conociendo que hay dos rectas paralelas (y sus vectores directores).

		\item  Es perpendicular a la recta $r: (x,y,z)=(2,-9,7)+t(1,-1,1)$ y pasa por el punto $P(3,1,-3)$.
		\answer Guía: Para definir un plano necesitamos un puntoo y una normal (ec. general). Investigar que relación tiene la normal de un plano con el vector director de una recta perpendicular a él.

		\item  Contiene a la recta $r: (x,y,z)=(5,1,0)+t(1,-3,1)$.

		\item Es perpendicular a los planos $\pi_1: 2x-3y+5z=2$ y $\pi_2: -x+3y+2z=1$ e incluye al punto $P(5,1,0)$.

		\item Es perpendicular al planos $\pi: 3x-2y+z=5$ y pasa por $P(4,1,5)$.

		\item  Contiene a la recta $r: (x,y,z)=(2,1,2)+t(0,4,1)$ y al punto $P(1,2,3)$

		\item  Contiene a las rectas $r_1: (x,y,z)=(13,-2,0)+t(-6,0,2)$ y $r_2: (x,y,z)=(-14,-8,16)+t(5,2,-4)$.

	\end{enumcols}


	\exercise Analizar la posición relativa entre los siguientes lugares geométricos. Según corresponda, calcular la distancia entre ellos o la intersección y el menor ángulo. 
	\begin{enumcols}

		\item $r_1: (x,y,z)=(13,-2,0)+t(-6,0,2)$ ~~y~~ $r_2: (x,y,z)=(-14,-8,16)+t(5,2,-4)$

		\item $r_1: (x,y,z)=(1,1,1)+t(1,2,3)$ ~~y~~ $r_2: (x,y,z)=(2,1,0)+t(3,8,13)$
		\answer $r_1 \cap r_2 = \{(5,9,13)\}$. Resolución por \href{https://youtu.be/KebOzsUUmq4?t=2304}{Mate316}.

		\item $L_1:\SEL{x=1-\lambda \\ y=-1 \\ z= -\lambda}$ ~~y~~ $L_2:x+2=-y=-(z-1)$
		\answer $L_1$ y $L_2$ son alabeadas. $d(L_1,L_2)=\f{2}{\sqrt{6}}$. Resolución por Álgebra Para Todos: \href{https://youtu.be/Xc7NzupOj5k}{parte 1} y \href{https://youtu.be/PKH0ODI8RQA}{parte 2}.

		\item $r_1: (x,y,z)=(5,4,-1)+t(-6,0,2)$ ~~y~~ $r_2: (x,y,z)=(2,2,-1)+t(-3,2,1)$
		\item $r_1: (x,y,z)=(4,0,2)+t(2,4,6)$ ~~y~~ $r_2: (x,y,z)=(9,10,15)+t(3,6,9)$
		\item $\pi_1: 2x-3y+5z=2$ ~~y~~ $\pi_2: -x+3y+2z=1$.
		\item $\pi_1: 2x-4y+8z=3$ ~~y~~ $\pi_2: 3x-6y+12z=2$.
		\item $\pi_1: 2x-4y+8z=3$ ~~y~~ $\pi_2: 6x-12y+24z=9$.
		\item $r_1: (x,y,z)=(4,0,2)+t(2,4,6)$ ~~y~~ $\pi_2: 6x-12y+24z=9$
		\item $\pi_1: 2x-4y+8z=3$ ~~y~~ $r_2: (x,y,z)=(2,2,-1)+t(-3,2,1)$

	\end{enumcols}


	\exercise Resolver los siguientes problemas integradores sobre la geometría en el espacio mediante la construcción de elementos auxiliares, el uso de propiedades y el análisis geométrico.
	\begin{enumcols}
		
		\item Hallar $k\in \R$ para que $L_1:\SEL{x+y+z=0 \\ 2x-ky=4}$ sea paralela a vector $\vec{v}=(1,1,0)$
		\answer $\nexists k\in \R$ que cumpla la condición. Resolución por \href{https://youtu.be/qIQflgvSxc4}{Álgebra Para Todos}.

		\item Identificar todos los valores de $k\in\R$ tales que la distancia de la recta $r:\SEL{x+y+z=1 \\ x-kz=0}$ al origen de coordenadas sea $1$.
		\answer $k=-1$. Resolución por \href{https://youtu.be/UNMMktIFFAE}{Álgebra Para Todos}.

		\item Dados los puntos $A(3,1,1)$ y $B(3,-2,4)$, y la recta $L: (x,y,z)=(1,-1,1)+t(1,1,0)$. Encontrar un punto $C\in L$ tal que el ángulo entre $\vect{AB}$ y $\vect{AC}$ sea $60\degs$.

		\item Calcular la distancia de la recta $L_1: (x,y,z) = (2,1,0) + \lambda (1,1,1)$ al plano $\pi$ que contiene al eje $z$ y que es paralelo a la recta $L_1$. Esquematizar la situación.
		\answer $\pi: x-y=0$ y $d(\pi,L_1)=\f{1}{\sqrt{2}}$. Resolución por Álgebra Para Todos: \href{https://youtu.be/PmysAG03Y-s}{parte 1} y \href{https://youtu.be/X81YN6615Yg}{parte 2}. 

		\item Obtener la ecuación del plano que contiene a la recta $s:\SEL{x=3 \\ y=2}$ y forma un ángulo de $30\degs$ con la recta $r: (x,y,z) = (0,0,0) t(-1,0,1)$. Esquematizar la situación.
		\answer Existen dos planos que cumples las condiciones: $\pi_1: x-y-1=0$ y $\pi_2: x+y-5=0$. Resolución por \href{https://youtu.be/Xw0cLEM9VK0}{Álgebra Para Todos}.

		\item Encontrar la proyección de $L: x-2=y-1=\f{-z-3}{2}$ sobre el plano $\pi: x+y+z=0$.
		\answer La proyección de $L$ sobre $\pi$ es la propia $L$, ya que $L \subseteq \pi$. Resolución por \href{https://youtu.be/l2aUePfNSQQ}{Álgebra Para Todos}.

		\item Dada la recta $L:\SEL{y=2x\\ z=1}$ y el plano $pl_{\alpha}: x+y+z=9$, determinar la ecuación del plano que contiene a la recta $L$ y es perpendicular a $pl_{\alpha}$. Luego, halle la proyección de la recta $L$ sobre $pl_{\alpha}$.
		\answer El plano es $pl_{\beta}: 2x -y -z +1 = 0$ y la proyección de la recta es $L': (x,y,z)=\left(\frac{8}{3},0,\frac{19}{3}\right)+k(0,1,-1)$. Resolución por Álgebra Para Todos: \href{https://youtu.be/aricPfApQqM}{parte 1}, \href{https://youtu.be/PD2a3LBVwvg}{parte 2} y \href{https://youtu.be/QNaua8cv-3w}{parte 3}.

		\item Calcular los valores de las constantes $a$ y $b$ para que la proyección de la recta $r:\SEL{x=-2t \\ y=1+at \\ z=bt}$ sobre el plano $\pi: x+2z=30$ resulte en un único punto y dar dicho punto.
		\answer Las constantes son $a=0$ y $b=-4$ y el punto proyección es $P(6,1,12)$. Resolución por \href{https://youtu.be/Q672TL02qKY}{Álgebra Para Todos}.

		\item Hallar la proyección del punto $A(1,1,0)$ sobre el plano que determinan las rectas $L_1: \f{x-1}{-1}=y-2=z$ y $L_2:\SEL{x=1-2\beta\\ 3+\beta\\ z=1+\beta}$. Esquematizar la situación.
		\answer El plano formado por las rectas es $\pi: y-z-2=0$ y la proyección es $A'\left(1, \frac{3}{2}, -\frac{1}{2}\right)$. Resolución por Álgebra Para Todos: \href{https://youtu.be/zR3gJD7uBwg}{parte 1} y \href{https://youtu.be/9uSJGXEd-z0}{parte 2}.

		\item Averiguar el valor del parámetro $a$ para que $L_1:\SEL{2x-4y+z+1=0 \\ x-2y+3z-2=0}$ ~~y~~ $L_2:\f{x-1}{a}=\f{y+2}{4}=z$ sean coplanares. Además encontrar todos los puntos $P$ pertenecientes al eje $y$ tal que la distancia al plano que contiene a $L_1$ y $L_2$ sea igual a $\sqrt{41}$
		\answer El parámetro es $a=2$ y los puntos son $P_1(0, -23, 0)$ y $P_2 (0,18,0)$. Resolución por Álgebra Para Todos: \href{https://youtu.be/P9trZC_oQSE}{parte 1}, \href{https://youtu.be/keQrj5XZTaw}{parte 2} y \href{https://youtu.be/0sS1fLfxRBU}{parte 3}.

		\item Sean $r: \SEL{-x+y-1=0 \\ x+tz+h=0}$ ~~y~~ $\pi: x+y+z+1=0$. Determinar los valores de $t$ y $h$ para los cuales $r\subseteq\pi$.
		\answer $t=\frac{1}{2}$ y $h=1$. Resolución por \href{https://youtu.be/Yrs27CRvPIM}{Álgebra Para Todos}. 

		\item Sean las rectas $r_1: x+1 = y+2 = \f{z}{3}$ ~~y ~~ $r_2:\SEL{x=1+2\lambda \\ y=\lambda \\ z=3-\lambda}$. Obtener las ecuaciones de todos los planos $\pi$ tales que $\pi \perp r_1$ y $d(\pi, r_2)=\sqrt{11}$
		\answer Hay dos planos posibles: $\pi_1: x+y+3z+1=0$ y $\pi_2: x+y+3z-21=0$. Resolución por \href{https://youtu.be/6aLjzBa_U6w}{Álgebra Para Todos}. 

		\item Determinar la ecuación de la recta que contiene al origen de coordenadas, es perpendicular a la recta $r: (x,y,z) = (-5, 0, -3) + \lambda(1,1,2)$ y es incidente a la recta $s: (x,y,z)=(0,1,2) + t (1,2,1)$.
		\answer $L: (x,y,z)=(0,0,0)+k(1,1,-1)$. Como la recta buscada $L$ para por el origen y es perpendicular a $r$, entonces $L$ está contenida en el plano $\pi_{aux}: x+y+2z=0$. La intersección $L \cap s$ se encuentra dentro de la intersección $\pi_{aux} \cap s$, que es el punto $A(-1,-1,1)$. Con los puntos $O$ y $A$ obtendo la ecuación de la recta $L: (x,y,z)=(0,0,0)+k \vect{AO}=(0,0,0)+k(1,1,-1)$. Resolución alternativa por \href{https://youtu.be/x4t4zb0T8Eg}{Álgebra Para Todos}. 

		\item Ubicar el punto de la recta $r:\SEL{x=1-t \\ y=3+t \\2+2t}$ cuya proyección sobre el plano $yz$ es $(0,1,-2)$.
		\answer $P(3,1,-2)$. Resolución por \href{https://youtu.be/GhhsZmKsetU}{Álgebra Para Todos}.

		\item Hallar todos los puntos $P$ del plano $xy$ tal que la distancia al plano $\alpha: 3x+4y-12=0$ es igual a $1$. Indicar que lugar geométrico representa	el conjunto de todos los puntos $P$ y esquematizar la situación.
		\answer Son dos rectas paralelas al eje $x$, $r_1: \SEL{x=0 \\ y=\frac{17}{3}\\ z=0}$ ~~y~~ $r_1: \SEL{x=0 \\ y=\frac{7}{3}\\ z=0}$. Resolución por \href{https://youtu.be/d8bTaE8wNRg}{Álgebra Para Todos}.


	\end{enumcols}


\end{enumerate}

\end{document}