\documentclass{template_practica}

\begin{document}

\practiceheader{Práctica 3: Vectores}{Comisión: Rodrigo Cossio-Pérez y Gabriel Romero}

\begin{enumerate}

	\exercise Definir los vectores a partir de sus puntos extremo y origen, y graficarlos desplazados al origen indicando su cuadrante/octante.
	\begin{enumcols}[2]
		
		\item $\vect{AB}$, con $A(1,2)$ y $B(3,4)$
		\answer $\vect{AB}= \vect{OB} - \vect{OA} = (3,4) - (1,2) = (2,2)$

		\item $\vec{v}$ es el vector con origen $C(3,1)$ y $D(0,4)$
		\answer $\vec{v}= \vect{OD} - \vect{OC} = (0,4) - (3,1)= (-3,3)$

		\item $\vect{EF}$ y $\vect{FE}$, con $E(-1,2)$ y $F(1,1)$
		\answer $\vect{EF} = \vect{OF} - \vect{OE} = (1,1) - (-1,2) = (2,-1)$. $\vect{FE} = \vect{OE} - \vect{OF} = (-1,2) - (1,1) = (-2,1)$

		\item $\vect{AB}$, con $A(1,2,3)$ y $B(1,4,4)$
		\answer $\vect{AB} = \vect{OB} - \vect{OA} = (1,4,4) - (1,2,3) = (0,2,1)$

		\item $\vec{u}$ que parte desde el origen al punto $D(2,-3,4)$
		\answer $\vec{u} = \vect{OD} = (2,-3,4)$

		\item $\vec{r}$ con origen $P_0(0,0,4)$ y $P_1(2,2,1)$
		\answer $\vec{r} = \vect{OP_1} - \vect{OP_0} = (2,2,1) - (0,0,4) = (2,2,-3)$
 
	\end{enumcols}


	\exercise Interpretar gráficamente las siguientes sumas y restas de vectores, acompañando el gráfico con el cálculo analítico.
	\begin{enumcols}[2]
		
		\item $(1,1)+(2,3)$
		\answer $(1,1)+(2,3) = (3,4)$

		\item $\vec{v} + \vec{u}$ con $\vec{v}=(-1,3)$ y $\vec{u}=(1,3)$
		\answer $\vec{v} + \vec{u} = (-1,3) + (1,3) = (0,6)$

		\item $(4,-1)+(3,3)+(-2,0)$
		\answer $(4,-1)+(3,3)+(-2,0) = (5,2)$

		\item $(3,1)-(2,4)$
		\answer $(3,1)-(2,4) = (1,-3)$

		\item $\vec{a} - \vec{b}$ y $\vec{b} - \vec{a}$ con $\vec{a}=(-1,2)$ y $\vec{b}=(1,2)$
		\answer $\vec{a} - \vec{b} = (-1,2) - (1,2) = (-2,0)$. $\vec{b} - \vec{a} = (1,2) - (-1,2) = (2,0)$

		\item $\left(\vec{v}+\vec{u}\right)-\vec{w}$ con $\vec{v}=(1,2)$, $\vec{u}=(2,1)$ y $\vec{w}=(1,1)$
		\answer $\left(\vec{v}+\vec{u}\right)-\vec{w} = (1,2)+(2,1)-(1,1) = (2,2)$

		\item $(2,0,0)+(0,3,0)+(0,0,1)$
		\answer $(2,0,0)+(0,3,0)+(0,0,1) = (2,3,1)$

		\item $(2,2,0)+(0,-1,1)$
		\answer $(2,2,0)+(0,-1,1) = (2,1,1)$

		\item $(1,-1,0)-(-2,-1,3)$
		\answer $(1,-1,0)-(-2,-1,3) = (3,0,-3)$

		\item $(0,4)+(2,0)+(1,-1)+(-1,-1)+(-2,0)+(2,-2)$
		\answer $(2,0)$

	\end{enumcols}


	\exercise Calcular el módulo y ángulo de los siguientes vectores de $\R^2$ y escribir sus coordenadas en forma polar.
	\begin{enumcols}[2]
		
		\item $(1,1)$
		\answer $|(1,1)| = \sqrt{1^2+1^2} = \sqrt{2}$. $\theta=\arctan\left(\f{1}{1}\right) = \arctan(1) = 45\degs$. $\SEL{v_x=\sqrt{2} \cos{45\degs} \\ v_y=\sqrt{2} \sin{45\degs} }$

		\item $(2,3)$
		\answer $|(2,3)| = \sqrt{2^2+3^2} = \sqrt{13}$. $\theta=\arctan\left(\f{3}{2}\right) \simeq 56.31\degs$.  $\SEL{v_x \simeq\sqrt{13} \cos{56\degs} \\ v_y\simeq\sqrt{13} \sin{56\degs} }$

		\item $(3,-4)$
		\answer $|(3,-4)| = \sqrt{3^2+(-4)^2} = \sqrt{25} = 5$. $\theta=\arctan\left(\f{-4}{3}\right) \simeq -53.13\degs$. $\SEL{v_x \simeq 5 \cos(-53\degs) \\ v_y\simeq5 \sin(-53\degs) }$

		\item $(0,5)$
		\answer $|(0,5)| = \sqrt{0^2+5^2} = \sqrt{25} = 5$. $\theta= 90\degs$. $\SEL{v_x=5 \cos{90\degs} \\ v_y=5 \sin{90\degs} }$

		\item $(-2,1)$
		\answer $|(-2,1)| = \sqrt{(-2)^2+1^2} = \sqrt{5}$. $\theta=\arctan\left(\f{1}{-2}\right) +180\degs \simeq 153.43\degs$. $\SEL{v_x \simeq \sqrt{5} \cos{153\degs} \\ v_y \simeq \sqrt{5} \sin{153\degs} }$

		\item $(-3,-4)$
		\answer $|(-3,-4)| = \sqrt{(-3)^2+(-4)^2} = \sqrt{25} = 5$. $\theta=\arctan\left(\f{-4}{-3}\right) +180\degs \simeq 233.13\degs$. $\SEL{v_x \simeq 5 \cos{233\degs} \\ v_y \simeq 5 \sin{233\degs} }$

		\item $(-4,0)$
		\answer $|(-4,0)| = \sqrt{(-4)^2+0^2} = \sqrt{16} = 4$. $\theta= 180\degs$. $\SEL{v_x = 4 \cos{180\degs} \\ v_y = 4 \sin{180\degs} }$

		\item $(0,0)$
		\answer $|(0,0)| = \sqrt{0^2+0^2} = \sqrt{0} = 0$. No se define un ángulo. $\SEL{v_x = 0 \\ v_y= 0}$

	\end{enumcols}


	\exercise Escribir los siguientes vectores de $\R^2$ en coordenadas rectangulares a partir de su ángulo y módulo
	\begin{enumcols}[2]
		
		\item $\left|\vec{v}\right|=5$ y $\theta=30\degs$
		\answer $\vec{v}=(5 \cos{30\degs}, 5 \sin{30\degs})=\left(\f{5\sqrt{3}}{2}, \f{5}{2}\right)$

		\item $\left|\vec{u}\right|=8$ y $\theta=60\degs$
		\answer $\vec{u}=(8 \cos{60\degs}, 8 \sin{60\degs})=( 4, 4\sqrt{3})$

		\item $\left|\vec{r}\right|=4$ y $\theta=180\degs$
		\answer $\vec{r}=(4 \cos{180\degs}, 4 \sin{180\degs})= (-4,0)$

		\item $\left|\vec{P}\right|=7$ y $\theta=300\degs$
		\answer $\vec{P}=(7 \cos{300\degs}, 7 \sin{300\degs})= \left(-\f{7}{2}, -\f{7\sqrt{3}}{2}\right)$

		\item $\left|\vec{w}\right|=3$ y $\theta=145\degs$
		\answer $\vec{w}=(3 \cos{145\degs}, 3 \sin{145\degs})= \left(-\f{3\sqrt{2}}{2}, \f{3\sqrt{2}}{2}\right)$

		\item $\left|\vec{F}\right|=1$ y $\theta=210\degs$
		\answer $\vec{F}=(\cos{210\degs}, \sin{210\degs})= \left(-\f{\sqrt{3}}{2}, -\f{1}{2}\right)$

		\item $\left|\vec{T}\right|=2$ y $\theta=0\degs$
		\answer $\vec{T}=(2 \cos{0\degs}, 2 \sin{0\degs})= (2,0)$
		
		\item $\left|\vec{L}\right|=6$ y $\theta=330\degs$
		\answer $\vec{L}=(6 \cos{330\degs}, 6 \sin{330\degs})= (3\sqrt{3}, -3)$

		\item $\left|\vec{a}\right|=5$ y $\theta=-45\degs$
		\answer $\vec{a}=(5 \cos{-45\degs}, 5 \sin{-45\degs})= \left(\f{5\sqrt{2}}{2}, -\f{5\sqrt{2}}{2}\right)$

		\item $\left|\vec{n}\right|=0$
		\answer $\vec{n}=(0, 0)$

	\end{enumcols}


	\exercise De los siguientes vectores de $\R^3$, calcular su módulo $|\vec{v}|$, coordenada radial $\rho$, ángulo azimutal $\phi$ y colatitud $\theta$. Luego, interpretarlos graficamente y escribirlos en coordenadas cilíndricas y esféricas.
	\begin{enumcols}[2]
		
		\item $(1,1,1)$
		\answer $|\vec{v}| = \sqrt{1^2+1^2+1^2} = \sqrt{3}$. $\rho = \sqrt{1^2+1^2} = \sqrt{2}$. $\phi = \arctan\left(\f{1}{1}\right) = 45\degs$. $\theta = \arccos\left(\f{1}{\sqrt{3}}\right) \simeq 54.74\degs$. \\ Cilíndricas: $\SEL{v_x=\sqrt{2} \cos{45\degs} \\ v_y=\sqrt{2} \sin{45\degs} \\ v_z=1 }$. Esféricas: $\SEL{v_x \simeq \sqrt{3} \sin{54\degs} \cos{45\degs} \\ v_y \simeq \sqrt{3} \sin{54\degs} \sin{45\degs} \\ v_z \simeq \sqrt{3} \cos{54\degs} }$ 

		\item $(2,3,4)$
		\answer $|\vec{v}| = \sqrt{2^2+3^2+4^2} = \sqrt{29}$. $\rho = \sqrt{2^2+3^2} = \sqrt{13}$. $\phi = \arctan\left(\f{3}{2}\right) \simeq 56.31\degs$. $\theta = \arccos\left(\f{4}{\sqrt{29}}\right) \simeq 42.03\degs$. \\ Cilíndricas: $\SEL{v_x \simeq \sqrt{13} \cos{56\degs} \\ v_y \simeq \sqrt{13} \sin{56\degs} \\ v_z=4 }$. Esféricas: $\SEL{v_x \simeq \sqrt{29} \sin{42\degs} \cos{56\degs} \\ v_y \simeq \sqrt{29} \sin{42\degs} \sin{56\degs} \\ v_z \simeq \sqrt{29} \cos{42\degs} }$

		\item $(3,-4,0)$
		\answer $|\vec{v}| = \sqrt{3^2+(-4)^2+0^2} = 5$. $\rho = \sqrt{3^2+(-4)^2} = 5$. $\phi = \arctan\left(\f{-4}{3}\right) \simeq -53.13\degs$. $\theta = \arccos\left(\f{0}{5}\right) = 90\degs$. \\ Cilíndricas: $\SEL{v_x \simeq 5 \cos(-53\degs) \\ v_y \simeq 5 \sin(-53\degs) \\ v_z=0 }$. Esféricas: $\SEL{v_x \simeq 5 \sin{90\degs} \cos(-53\degs) \\ v_y \simeq 5 \sin{90\degs} \sin(-53\degs) \\ v_z \simeq 5 \cos{90\degs} }$

		\item $(0,5,0)$
		\answer $|\vec{v}| = \sqrt{0^2+5^2+0^2} = 5$. $\rho = \sqrt{0^2+5^2} = 5$. $\phi = 90\degs$. $\theta = \arccos\left(\f{0}{5}\right) = 90\degs$. \\ Cilíndricas: $\SEL{v_x=5 \cos{90\degs} \\ v_y=5 \sin{90\degs} \\ v_z=0 }$. Esféricas: $\SEL{v_x \simeq 5 \sin{90\degs} \cos{90\degs} \\ v_y \simeq 5 \sin{90\degs} \sin{90\degs} \\ v_z \simeq 5 \cos{90\degs} }$

	\end{enumcols}


	\exercise Escribir los siguientes vectores de $\R^3$ en coordenadas rectangulares a partir de su módulo $|\vec{v}|$, coordenada radial $\rho$, cota $z$, ángulo azimutal $\phi$ y/o colatitud $\theta$.
	\begin{enumcols}[2]
		
		\item $\rho=3$, $\phi=45\degs$ y $z=2$
		\answer $\vec{v} = (3 \cos{45\degs}, 3 \sin{45\degs}, 2)= \left(\f{3\sqrt{2}}{2}, \f{3\sqrt{2}}{2}, 2 \right)$

		\item $\rho=2$, $\phi=180\degs$ y $z=4$
		\answer $\vec{v} = (2 \cos{180\degs}, 2 \sin{180\degs}, 4)= (-2, 0, 4)$

		\item $\rho=1$, $\phi=300\degs$ y $z=-2$
		\answer $\vec{v} = (1 \cos{300\degs}, 1 \sin{300\degs}, -2)= \left(\f{1}{2}, -\f{\sqrt{3}}{2}, -2 \right)$

		\item $\rho=4$, $\phi=0\degs$ y $z=0$
		\answer $\vec{v} = (4 \cos{0\degs}, 4 \sin{0\degs}, 0)= (4, 0, 0)$

		\item $\rho=0$ y $z=3$
		\answer $\vec{v}=(0,0,3)$

		\item $\left|\vec{v}\right|=5$, $\phi=45\degs$ y $\theta=30\degs$
		\answer $\vec{v} = (5 \sin{30\degs} \cos{45\degs}, 5 \sin{30\degs} \sin{45\degs}, 5 \cos{30\degs})= \left(\f{5\sqrt{2}}{4}, \f{5\sqrt{2}}{4}, \f{5\sqrt{3}}{2} \right)$

		\item $\left|\vec{T}\right|=3$, $\phi=270\degs$ y $\theta=90\degs$
		\answer $\vec{T} = (3 \sin{90\degs} \cos{270\degs}, 3 \sin{90\degs} \sin{270\degs}, 3 \cos{90\degs})= (0, -3, 0)$

		\item $\left|\vec{d}\right|=2$ y $\theta=0\degs$
		\answer $\vec{d} = (0, 0, 2)$

		\item $\left|\vec{F}\right|=1$ y $\theta=180\degs$
		\answer $\vec{F} = (0, 0, -1)$

		\item $\left|\vec{d}\right|=0$
		\answer $\vec{d} = (0, 0, 0)$

	\end{enumcols}


	\exercise Resolver los siguientes ejercicios sobre geometría
	\begin{enumcols}
		
		\item Calcular la distancia entre los puntos $P(-1,2,7)$ y $Q(3,0,1)$
		\answer $d(P,Q) = \left| \vect{PQ} \right| = \left| (4,-2,-6) \right| = \sqrt{4^2+(-2)^2+(-6)^2} = \sqrt{56} \simeq 7.4833$

		\item Dado $\vect{PQ}=(5,7)$ y $P(-1,-1)$, calcular $Q$.
		\answer $\vect{PQ} = \vect{OQ} - \vect{OP}$, por lo que $\vect{OQ}=\vect{OP}+\vect{PQ}=(-1,-1)+(5,7)=(4,6)$. El punto es $Q(4,6)$

		\item Dado $\vect{AB}=(4,1)$ y $B(1,2)$, calcular $A$
		\answer $\vect{AB} = \vect{OB} - \vect{OA}$, por lo que $\vect{OA}=\vect{OB}-\vect{AB}=(1,2)-(4,1)=(-3,1)$. El punto es $A(-3,1)$

		\item Indicar si la distancia entre $R(1,-4,0)$ y $S(4,1,2)$ es menor a $10$ unidades
		\answer $d(R,S) = \left| \vect{RS} \right| = \left| (3,5,2) \right| = \sqrt{3^2+5^2+2^2} = \sqrt{38} \simeq 6.1644$. La distancia es menor a $10$ unidades.

		\item El triangulo $ABC$ tiene vértices $A(1,2,3)$, $B(3,3,2)$ y $C(1,0,5)$. Calcular el perímetro del triángulo.
		\answer El perímetro es la suma de los lados $\overline{AB}$, $\overline{BC}$ y $\overline{CA}$. Calculamos cada lado: \\ $\overline{AB} = \left| \vect{AB}\right| = \left| (2,1,-1) \right| = \sqrt{2^2+1^2+(-1)^2} = \sqrt{6}$. \\ $\overline{BC} = \left| \vect{BC}\right| = \left| (-2,-3,3) \right| = \sqrt{(-2)^2+(-3)^2+3^2} = \sqrt{22}$. \\ $\overline{CA} = \left| \vect{CA}\right| = \left| (0,-2,2) \right| = \sqrt{0^2+(-2)^2+2^2} = \sqrt{8}$. \\El perímetro es $\sqrt{6}+\sqrt{22}+\sqrt{8} \simeq 9.9683$

		\item La suma de los vectores $\vect{v_1}$ , $\vect{v_2}=(5,1)$ y $\vect{v_3} = (-5,1)$ es el vector nulo. Calcular $\vect{v_1}$ e interpretar gráficamente
		\answer Como $\vect{v_1}+\vect{v_2}+\vect{v_3}=\vect{0}$, despejamos $\vect{v_1} = \vect{0} -\vect{v_2} - \vect{v_3} = (0,0) - (5,1) - (-5,1) = (0,-2)$.

		\item Se sabe que $\vect{u_1} + \vect{u_2} = \vect{u_3}$ y que $\vect{u_1} = (1,1,0)$ y $\vect{u_3}=(0,2,0)$. Calcular $\vect{u_2}$ e interpretar gráficamente.
		\answer Como $\vect{u_1} + \vect{u_2} = \vect{u_3}$, despejamos $\vect{u_2} = \vect{u_3} - \vect{u_1} = (0,2,0) - (1,1,0) = (-1,1,0)$

		\item Dados $\vec{a},\vec{b}$ en $\R^2$, se sabe que $\vec{a}+\vec{b}=(5,4)$ y que $\vec{a}-\vec{b}=(1,-2)$. Calcular $\vec{a}$ y $\vec{b}$ y comprobar el resultado gráficamente.
		\answer Como $\vec{a}+\vec{b}=(5,4)$ y $\vec{a}-\vec{b}=(1,-2)$, sumando ambas ecuaciones obtenemos $2\vec{a}=(6,2)$, por lo que $\vec{a}=(3,1)$. Restando la primera ecuación con la segunda obtenemos $2\vec{b}=(4,6)$, por lo que $\vec{b}=(2,3)$.

		\item Determinar $D$ para que $\vect{AB}$ sea equivalente a $\vect{CD}$ con $A(5,-1,3)$, $B(2,3,-2)$, $C(-5,8,-5)$. Interpretar gráficamente.
		\answer $\vect{AB} = \vect{OB} - \vect{OA} = (2,3,-2) - (5,-1,3) = (-3,4,-5)$. Por lo tanto $\vect{CD} = \vect{AB} = (-3,4,-5)$. Además $ \vect{CD}= \vect{OD} - \vect{OC}$ por lo que $\vect{OD} = \vect{OC} + \vect{CD} = (-5,8,-5) + (-3,4,-5) = (-8,12,-10)$. El punto es $D(-8,12,-10)$

		\item Encontrar las coordenadas del punto medio entre $A(7,5)$ y $B(1,2)$
		\answer Las coordenadas del punto medio son $\vect{OC}=\f{1}{2}\left(\vect{OA}+\vect{OB}\right) = \left(\f{7+1}{2}, \f{5+2}{2}\right) = \left(4,\f{7}{2}\right)$. El punto es $C\left(4,\f{7}{2}\right)$.

		\item Encontrar las coordenadas el punto medio entre $M(0,-1,1)$ y $N(2,2,-1)$
		\answer El punto medio es $\vect{OL}=\f{1}{2}\left(\vect{OM}+\vect{ON}\right) = \left(\f{0+2}{2}, \f{-1+2}{2}, \f{1-1}{2}\right) = \left(1,\f{1}{2},0\right)$. El punto es $L\left(1,\f{1}{2},0\right)$.

		\item Sean $A(2,-3)$, $B(-3,5)$ y $C(-1,5)$ y $D$ los vértices de un paralelogramo, donde $D$ es el vértice opuesto a $A$, hallar $D$ e interpretar geométricamente.
		\answer El centro del paralelogramo es el punto medio entre $B$ y $C$, por lo que el centro es $E(-2,5)$. El vector $\vect{AE}=(-4,8)$ es equivalente a $\vect{ED}$, por lo que $\vect{OD} = \vect{OE} + \vect{ED} = (-2,5) + (-4,8) = (-6,13)$. El punto es $D(-6,13)$

		\item Dado el vector $\vec{v}=(k,k+1)$, calcular $k$ para que $|\vec{v}|=5$ y dar las coordenadas del vector $\vec{v}$.
		\answer Directamente conviene plantear $|\vec{v}|^2$ en vez de $|\vec{v}|$. $|\vec{v}|^2=2k^2+2k+1=5^2$, es decir, $2k^2+2k-24=0$. Resolviendo la ecuación obtenemos $k_1=3$ o $k_2=-4$. El vector es $\vec{v}_1=(3,4)$ o $\vec{v}_2=(-4,3)$

		\item Hallar el vector $\vec{v}_{\theta}$ de $\R^2$ que tiene módulo $1$ y ángulo $\theta$. Luego, calcular $\vec{v}_{30 \degs}$, $\vec{v}_{60 \degs}$, $\vec{v}_{90 \degs}$, $\vec{v}_{135 \degs}$ y $\vec{v}_{300 \degs}$.
		\answer Considerando las coordenadas polares se obtiene $\vec{v}=(\cos{\theta}, \sin{\theta})$. Por lo tanto $\vec{v}_{30 \degs}=\left(\f{\sqrt{3}}{2},\f{1}{2}\right)$, $\vec{v}_{60 \degs}=\left(\f{1}{2},\f{\sqrt{3}}{2}\right)$, $\vec{v}_{90 \degs}=\left(0,1\right)$, $\vec{v}_{135 \degs}=\left(\f{\sqrt{2}}{2},\f{\sqrt{2}}{2}\right)$ y $\vec{v}_{300 \degs}=\left(\f{1}{2},-\f{\sqrt{3}}{2}\right)$

		\item Dar las coordenadas rectangulars del vector $\vec{v}$ sabiendo que tiene módulo $20$ y se cuenta con algunos ángulos que definen su dirección: \\ \img{4cm}{img/vec_r3_angles.png}
		\answer Calculamos la coordenada radial $\rho = |\vec{v}| \sin{30 \degs} = 5$ y la coordenada $v_z = |\vec{v}| \cos{30 \degs} = 10\sqrt{3}$. Luego, oservamos que el ángulo azimutal debe ser $\phi=135\degs + 90 \degs=225\degs$. Con lo que calculamos $v_x=\rho . \cos{225\degs}=-5\sqrt{2}$ y $v_y=\rho .\sin{225\degs}=-5\sqrt{2}$. Finalmente, $\vec{v}=\left(-5\sqrt{2},-5\sqrt{2},10\sqrt{3}\right)$

	\end{enumcols}


	\exercise Obtener los siguientes vectores definidos a partir del producto de vector con número escalar e interpretarlos gráficamente
	\begin{enumcols}
		\item $3 (3,2)$
		\answer $3 (3,2) = (9,6)$

		\item $-2 \vec{v}$ ~con $\vec{v}=(4,5)$
		\answer $-2 \vec{v} = -2 (4,5) = (-8,-10)$

		\item $3 (1,1,4) - 2(0,0,5)$
		\answer $3 (1,1,4) - 2(0,0,5) = (3,3,12) - (0,0,10) = (3,3,2)$

		\item $4 \vec{v} -2 \vec{u} + \vec{w}$ ~con $\vec{v}=(1,0,3)$, $\vec{u}=(-2,1,0)$ y $\vec{w}=(3,-3,3)$
		\answer $4 \vec{v} -2 \vec{u} + \vec{w} = 4(1,0,3) -2(-2,1,0) + (3,-3,3) = (4,0,12) - (-4,2,0) + (3,-3,3) = (11,-5,15)$

		\item $3 \vect{PQ} + 2 \vect{PR}$ ~con los puntos $P(1,2)$, $Q(2,3)$ y $R(1,4)$
		\answer $3 \vect{PQ} + 2 \vect{PR} = 3(1,1) + 2(0,2) = (3,3) + (0,4) = (3,7)$

		\item $-2 \vec{r} + 3 \vec{s}$ ~donde $\vec{r}=\vec{v}+\vec{u}$ ~~y~~ $\vec{s}=2\vec{u}$, con $\vec{v}=(1,2)$ y $\vec{u}=(3,4)$
		\answer $-2\vec{r} +3 \vec{s} = -2 (\vec{v}+\vec{u}) +3 (2 \vec{u}) = -2 \vec{v}-2 \vec{u} + 6\vec{u} = -2 \vec{v}+ 4\vec{u} = -2(1,2)+4(3,4)= (10,12)$

		\item $4 \ivec - 5 \jvec$
		\answer $4(1,0)-5(0,1)=(4,-5)$

		\item $-2 \ivec +0 \jvec + 7 \kvec$
		\answer $-2(1,0)+0(0,1)+7(0,0)=(-2,0,7)$

		\item $\f{1}{3} \vect{F} + 2\vect{T} - \f{5}{2}\vect{N}$ ~con $\vect{F}=(-3,2)$,~~ $\vect{T}=\SEL{ \left|\vect{T}\right| = 5 \\ \theta_{\vec{T}} = 30\degs}$ ~~y~~~ $\vect{N}=3\ivec$
		\answer Calculamos $\vect{T}=\left(5\cos{30\degs},5\sin{30\degs}\right)=\left(\f{5\sqrt{3}}{2},\f{5}{2}\right)$. Luego hacemos las operaciones \\ $\f{1}{3} (-3,2) + 2\left(\f{5\sqrt{3}}{2},\f{5}{2}\right)- \f{5}{2}(1,0)=\left(-1,\f{2}{3}\right) + \left(5\sqrt{3},5\right)- \left(\f{5}{2},0\right) = \left(-\f{7}{2}+5\sqrt{3}, \f{17}{3}\right) \simeq \left(5.1603, 5.6667\right)$

		\item $2(t,t+1,0) - 3(t,5,1)$ ~con $t\in\R$
		\answer $2(t,t+1,0) - 3(t,5,1) = (2t,2t+2,0) - (3t,15,3) = (-t,2t-13,-3)$

		\item $m(0,0,1)$ ~con $m\in\R$
		\answer $m(0,0,1) = (0,0,m)$

		\item $a(1,0,0) + b (0,1,0)$ ~con $a,b \in \R$
		\answer $a(1,0,0) + b (0,1,0) = (a,0,0) + (0,b,0) = (a,b,0)$

		\item $\f{1}{|\vec{v}|}~ \vec{v}$ ~con $\vec{v}=(3,-4)$
		\answer $\f{1}{|\vec{v}|}~ \vec{v} = \f{1}{\sqrt{3^2+(-4)^2}}~ (3,-4) = \f{1}{5}~ (3,-4) = \left(\f{3}{5},-\f{4}{5}\right)$

		\item $5\vec{v} +7 \vec{w}$ ~con $\vec{v}=\SEL{ \rho_{\vec{v}} = 4 \\ \theta_{\vec{v}} = 45\degs\\ v_z = 1 }$ ~~y~~ $\vec{w}=\SEL{ \rho_{\vec{w}} = 10 \\ \theta_{\vec{w}} = 210\degs\\ w_z = -1 }$
		\answer Convertimos a coordenadas cartesianas: $\vec{v}=(4\cos{45\degs},4\sin{45\degs},1)=(2\sqrt{2},2\sqrt{2},1)$ \\ y~ $\vec{w}=(10\cos{210\degs},10\sin{210\degs},-1)=(5\sqrt{3},5,-1)$. \\Por lo tanto, $5\vec{v} +7 \vec{w} = 5(2\sqrt{2},~2\sqrt{2},~1) +7 (5\sqrt{3},~5,~-1) = (10\sqrt{2}+35\sqrt{3}, ~~10\sqrt{2}+5, ~~0) \simeq (74.76, 19.14, 0)$
	\end{enumcols}


	\exercise Indicar si los siguientes pares de vectores son o no paralelos 
	\begin{enumcols}[2]
		\item $(1,2)$ y $(7,14)$
		\answer Son paralelos, ya que $(7,14)=7(1,2)$

		\item $(3,-1)$ y $(-6,-1)$
		\answer No son paralelos ya que $\f{-6}{3}\neq\f{-1}{-1}$

		\item $(5,-1)$ y $\left(-2,\f{2}{5}\right)$
		\answer Son paralelos, ya que $\left(-2,\f{2}{5}\right)=-\f{2}{5}(5,-1)$

		\item $\left(2,-\f{3}{4},\f{1}{2}\right)$ y $\left(2,-\f{3}{4},\f{1}{2}\right)$
		\answer Son paralelos, ya que $\left(2,-\f{3}{4},\f{1}{2}\right)=1\left(2,-\f{3}{4},\f{1}{2}\right)$

		\item $\left(1,4,1\right)$ y $\left(2,8,0\right)$
		\answer No son paralelos, ya que $\f{2}{1}=\f{8}{2}$ pero $\f{8}{2}\neq\f{0}{1}$.

		\item $\left(1,2,3\right)$ y $\left(-2,-4,-6\right)$
		\answer Son paralelos, ya que $\left(-2,-4,-6\right)=-2\left(1,2,3\right)$

		\item $(1,1,2)$ con $\vec{w}=\SEL{ \rho_w = \sqrt{2} \\ \theta_w = 225\degs\\ w_z = -2 }$
		\answer Tenemos que $\vec{w}=(\sqrt{2}\cos{225\degs},\sqrt{2}\sin{225\degs},-2)=(-1,-1,-2)$. Por lo que son paralelos, ya que $(-1,-1,-2)=-1(1,1,2)$

		\item $(4,-1)$ con $\left(2k,-\f{k}{2}\right)$ con $k\in\R-\{0\}$
		\answer Son paralelos, ya que $\left(2k,-\f{k}{2}\right) = \f{k}{2} (4,-1)=$. No conocemos cuanto vale $\f{k}{2}$ pero sabemos que es un número real distinto de cero.

		\item $(7,2,1)$ y su versor asociado 
		\answer Son paralelos, todo vector es paralelo a su versor asociado ya que $\vec{v} = k \vers{v}$ con $k=|\vec{v}|$.
	\end{enumcols}



	\exercise Proponer un vector que cumpla con las siguientes condiciones
	\begin{enumcols}
		\item Tiene igual dirección que $\vec{v}=(-4,1,2)$
		\answer Una opcion es $\vec{u}=2\vec{v}=(-8,2,4)$.

		\item Es paralelo a $\vec{u}=(1,2,3)$ y módulo menor que el de $\vec{u}$
		\answer Una opción es $\vec{v}=\f{1}{2}\vec{u}=\left(\f{1}{2},1,\f{3}{2}\right)$. $\vec{u}$ tiene módulo $\sqrt{14}$ mientras que $\vec{v}$ tiene módulo $\f{\sqrt{14}}{2}$.

		\item Tiene igual dirección que $\vec{s}=(1,2)$, módulo mayor que $\vec{s}$ y está en el 3er cuadrante
		\answer Una opción es $\vec{v}=-10\vec{s}=(-10,-20)$ que está en el 3er cuadránte y tiene módulo $\sqrt{500}$, que es mayor que $|\vec{s}|=\sqrt{5}$

		\item Es paralelo a $\vec{w}=(1,2)$ y tiene módulo $5$
		\answer Planteamos $\vec{v}=k\vec{w}=(k,2k)$ que tiene módulo $|\vec{v}|=\sqrt{k^2+(2k)^2}=\sqrt{5k^2}=5$. Por lo que despejamos $k=\sqrt{5}$ y $k=-\sqrt{5}$. Elegimos uno cualquiera (porque pide proponer uno), por ejemplo $k=\sqrt{5}$ que da el $\vec{v}=(\sqrt{5},2\sqrt{5})$. 

		\item Tiene igual dirección que $\vec{r}=(1,2,3)$ y mide $1$
		\answer Una opción es el versor de $\vec{r}$ que cumple con esas características, $\vers{r}=\f{1}{|\vec{r}|} \vec{r}= \f{1}{\sqrt{14}} (1,2,3) = \left( \f{1}{\sqrt{14}}, \f{2}{\sqrt{14}}, \f{3}{\sqrt{14}} \right)$

		\item Tiene la misma dirección que $\vec{L}=\SEL{ \rho_{\vec{L}} = 5 \\ \theta_{\vec{L}} = 60\degs\\ L_z = 2 }$ pero módulo 1
		\answer Calculamos $\vec{L}$ en coordenadas rectangulares: $\vec{L}=(5\cos{60\degs},5\sin{60\degs},2)=\left(\f{5}{2}, \f{5\sqrt{3}}{2}, 2 \right)$. Calculamos $|\vec{L}|=\sqrt{\left(\f{5}{2}\right)^2+\left(\f{5\sqrt{3}}{2}\right)^2+2^2}=\sqrt{\f{25}{4}+\f{75}{4}+4}=\sqrt{29}$. Finalmente, obtenemos el versor, que cumple con las condiciones: $\vers{L}=\f{1}{\sqrt{29}}\left(\f{5}{2}, \f{5\sqrt{3}}{2}, 2 \right)=\left(\f{5}{2\sqrt{29}}, \f{5}{2}\sqrt{\f{3}{29}}, \f{2}{\sqrt{29}} \right)$.

		\item Es unitario y paralelo a $\vect{PQ}$ con $P(1,7,5)$ y $Q(0,0,2)$
		\answer El versor asociado a $\vect{PQ}=(-1,-7,-3)$ cumple las condiciones y es $\f{1}{\left|\vect{PQ}\right|}\vect{PQ}=\f{1}{\sqrt{75}}(-1,-7,-3)=\left(\f{-1}{\sqrt{75}},\f{-7}{\sqrt{75}},\f{-3}{\sqrt{75}}\right)$.

		\item Es el vector unitario de $\vec{v}=(5,3)$
		\answer El versor asociado a $\vec{v}$ es $\f{1}{|\vec{v}|}\vec{v}=\f{1}{\sqrt{34}}(5,3)=\left(\f{5}{\sqrt{34}},\f{3}{\sqrt{34}}\right)$

		\item Es el versor asociado a $\vec{u}=(1,0,4)$
		\answer El versor asociado a $\vec{u}$ es $\vers{u}=\f{1}{|\vec{u}|}\vec{u}=\f{1}{\sqrt{17}}(1,0,4)=\left(\f{1}{\sqrt{17}},0,\f{4}{\sqrt{17}}\right)$

		\item Es el vector unitario asociado a $\vec{v}=(m,0,-m)$ con $m\in\R$
		\answer El versor asociado a $\vec{v}$ es $\vers{v}=\f{1}{|\vec{v}|}\vec{v}=\f{1}{\sqrt{m^2+(-m)^2}}(m,0,-m)=\left(\f{m}{\sqrt{2m^2}},0,\f{-m}{\sqrt{2m^2}}\right)$

		\item Tiene la misma dirección que $\vec{r}=\SEL{ |\vec{r}| = 3 \\ \theta_{\vec{r}} = 0\degs\\ \phi_{\vec{r}} = 45\degs }$ pero módulo 1
		\answer El versor $\vers{r}$ cumple las condiciones. En cordenadas esféricas, $\vers{r}=\SEL{ |\vers{r}| = 1 \\ \theta_{\vers{r}} = 0\degs\\ \phi_{\vers{r}} = 45\degs }$
	\end{enumcols}


	\exercise Indicar si los siguientes vectores son perpendiculares, si forman un ángulo agudo o si forman uno obtuso. 
	\begin{enumcols}[2]
		\item $(5,-1)$ y $(2,3)$
		\answer Como $(5,-1) \cdot (2,3) = 10 - 3 = 7$, los vectores forman un ángulo agudo.

		\item $(3,-5)$ y $(10,6)$
		\answer Como $(3,-5) \cdot (10,6) = 30 - 30 = 0$, los vectores son perpendiculares.

		\item $(-3,5,0)$ y $(4,5,-3)$
		\answer Como $(-3,5,0) \cdot (4,5,-3) = -12 + 25 + 0 = 13$, los vectores forman un ángulo agudo.

		\item $(0,1)$ y $(3,-1)$ 
		\answer Como $(0,1) \cdot (3,-1) = 0 - 1 = -1$, los vectores forman un ángulo obtuso.

		\item $(2,3,1)$ y $(5,3,-4)$
		\answer Como $(2,3,1) \cdot (5,3,-4) = 10 + 9 - 4 = 15$, los vectores forman un ángulo agudo.

		\item $\vec{w}=(3,-4,1)$ y el versor de $\vec{u}=(-2,6,0)$
		\answer Se puede utilizar $(-2,6,0)$ en vez de su versor, ya que tienen la misma dirección y sentido. \\ Como $(3,-4,1) \cdot (-2,6,0) = -6 - 24 + 0 = -30$, los vectores $\vec{w}$ y $\vers{u}$ forman un ángulo obtuso.

		\item $(1,2,3)$ y $(2,4,6)$
		\answer Como $(1,2,3) \cdot (2,4,6) = 2 + 8 + 18 = 28$, los vectores forman un ángulo agudo.

		\item $(-1,5)$ y $\vec{v}=\SEL{|\vec{v}| = 3 \\ \theta_v = -60\degs }$ 
		\answer Calculamos ls coordenadas rectangulares de $\vec{v} = \left(3\cos(-60\degs),3\sin(-60\degs)\right)= \left(\f{3}{2},-\f{3\sqrt{3}}{2}\right)$. El producto escalar es $(-1,5) \cdot \left(\f{3}{2},-\f{3\sqrt{3}}{2}\right) = -\f{3}{2}-\f{15\sqrt{3}}{2}=-\f{3+15\sqrt{3}}{2} < 0$, por lo que los vectores forman un ángulo obtuso.

		\item $(-2,7,1)$ con $(k,0,2k)$ con $k\in(0;+\infty)$
		\answer En lugar de $(k,0,2k)$, utilizamos el vector $(1,0,2)$ que tiene la misma dirección y sentido (ya que $k$ es positivo). Como $(-2,7,1) \cdot (1,0,2) = -2 + 0 + 2 = 0$, los vectores son perpendiculares.

	\end{enumcols}


	\exercise Indicar si los siguientes vectores son paralelos, perpendiculares o forman un ángulo agudo/obtuso. En caso de formar un ángulo agudo/obtuso, calcularlo. Y en caso de ser paralelos, indicar si tienen el mismo sentido o sentido contrario.
	\begin{enumcols}[2]
		\item $(-3,2)$ y $(1,4)$
		\answer El ángulo entre los vectores es $\alpha=\arccos\left(\f{\left(-3,2\right)\cdot\left(1,4\right)}{\left|\left(-3,2\right)\right|.\left|\left(1,4\right)\right|}\right)=\arccos\left(\f{5}{\sqrt{13}\sqrt{17}}\right)\simeq70.35\degs$

		\item $(1,2)$ y $\left(-\f{1}{2},-1\right)$
		\answer El ángulo entre los vectores es $\alpha=\arccos\left(\f{\left(1,2\right)\cdot\left(-\f{1}{2},-1\right)}{\left|\left(1,2\right)\right|.\left|\left(-\f{1}{2},-1\right)\right|}\right)=\arccos\left(\f{-\frac{5}{2}}{\sqrt{5}\frac{\sqrt{5}}{2}}\right)=\arccos(-1)=180\degs$. Los vectores son paralelos de sentido opuesto.

		\item $(5,-3)$ y $(3,5)$
		\answer Al calcular el producto escalar vemos que $(5,-3)\cdot(3,5)=15-15=0$. Los vectores son perpendiculares.

		\item $(1,-5)$ y $\ivec$
		\answer El ángulo entre los vectores es $\alpha=\arccos\left(\f{\left(1,-5\right)\cdot\ivec}{\left|\left(1,-5\right)\right|.\left|\ivec\right|}\right)=\arccos\left(\f{1}{\sqrt{26}.1}\right)\simeq78.69\degs$
		
		\item $(3,-1,0)$ y $(9,-3,0)$
		\answer El ángulo entre los vectores es $\alpha=\arccos\left(\f{\left(3,-1,0\right)\cdot\left(9,-3,0\right)}{\left|\left(3,-1,0\right)\right|.\left|\left(9,-3,0\right)\right|}\right)=\arccos\left(\f{30}{\sqrt{10}\sqrt{90}}\right)=\arccos(1)=0\degs$. Los vectores son paralelos con el mismo sentido.

		\item $(3,-4,1)$ y $\kvec$
		\answer El ángulo entre los vectores es $\alpha=\arccos\left(\f{\left(3,-4,1\right)\cdot\kvec}{\left|\left(3,-4,1\right)\right|.\left|\kvec\right|}\right)=\arccos\left(\f{1}{\sqrt{26}.1}\right)\simeq78.69\degs$

		\item $5\ivec$ y $-2\kvec$
		\answer Al calcular el producto escalar vemos que $(5\ivec)\cdot(-2\kvec)=-10(\ivec \cdot \kvec)=-10.~0=0$ Los vectores son perpendiculares.

		\item $(1,-7,3)$ y $(2,4,0)$
		\answer El ángulo entre los vectores es $\alpha=\arccos\left(\f{\left(1,-7,3\right)\cdot\left(2,4,0\right)}{\left|\left(1,-7,3\right)\right|.\left|\left(2,4,0\right)\right|}\right)=\arccos\left(\f{-26}{\sqrt{59}\sqrt{20}}\right)\simeq139.2\degs$

		\item Los vectores del plano $\vec{v}=\SEL{|\vec{v}| = 10 \\ \theta_{\vec{v}} = 135\degs }$ ~~y~~ $\vec{u}=\SEL{|\vec{u}| = 4 \\ \theta_{\vec{u}} = -45\degs }$ 
		\answer El ángulo entre los vectores es $\alpha=\theta_{\vec{v}} - \theta_{\vec{u}}= 135\degs - (-45\degs) = 180\degs$. Los vectores son paralelos de sentido opuesto.

		\item $(1,4)$ y $\vec{p}=\SEL{|\vec{p}| = 6 \\ \theta_{\vec{p}} = 270\degs }$
		\answer Obtenemos $\vec{p}=(6\cos{270\degs, 6\sin{270 \degs}})=(0,-6)$. El ángulo entre los vectores es $\alpha=\arccos\left(\f{\left(1,4\right)\cdot\left(0,-6\right)}{\left|\left(1,4\right)\right|.\left|\left(0,-6\right)\right|}\right)=\arccos\left(\f{-24}{\sqrt{17}.6}\right)\simeq 166\degs$

	\end{enumcols}


	\exercise Si es posible, escribir el vector indicado como combinación lineal de los otros dados. En caso de que halla infinitas posibilidades, indicar la expresión analítica que cumplen.
	\begin{enumcols}
		\item $\vec{v}=(4,0,-2)$ como combinación lineal de los vectores $\ivec$, $\jvec$ y $\kvec$
		\answer $\vec{v}=4\ivec + 0\jvec -2 \kvec$

		\item $\vec{v} = (5,12)$ de la forma $\alpha \vec{u}+ \beta \vec{w}$ con $\vec{u}=(1,2)$, $\vec{w}=(-2,3)$ ~y~ $\alpha,\beta \in \R$
		\answer Al plantear $(5,12)=\alpha(1,2) + \beta(-2,3)$ obtenemos el sistema $\SEL{\alpha-2\beta=5 \\ 2\alpha+3\beta=12}$. \\ Al resolverlo da $\alpha=\f{39}{7}$ y $\beta=\f{2}{7}$. Por lo tanto $\vec{v}=\f{39}{7}(1,2)+\f{2}{7}(-2,3)$

		\item $\vec{t}=(4,0,-5)$ como combinación lineal de $(2,-5,0)$, $(1,1,0)$ y $(0,0,2)$
		\answer Al plantear $(4,0,-5)=\alpha (2,-5,0) + \beta (0,1,0) + \gamma(0,0,2)$ obtenemos el sistema $\SEL{2\alpha=4\\ -5\alpha+\beta=0 \\ \gamma=-5}$. \\ Al resolverlo obtenemos $\alpha=2$, $\beta=10$ y $\gamma=-5$. Por lo tanto $\vec{t}=2(2,-5,0)+10(0,1,0)-5(0,0,2)$

		\item $(4,-12)$ a partir de $\{(1,-3),(-3,9)\}$
		\answer Planteamos $(4,-12)=a(1,-3)+b(-3,9)$ y obtenemos el sistema $\SEL{a-3b=4 \\ -3a+9b=-12}$. De la primera ecuación obtenemos $a=4+3b$ y sustituyendo en la segunda ecuación obtenemos que se comprueba. Es un sistema con infinitas soluciones donde $a$ depende de $b$ según la ecuación $a=4+3b$. Por lo tanto, $(4,-12)=(4+3b)(1,-3)+b(-3,9)$ donde $b\in\R$ es un parámetro. Por ejemplo si $b=1$, $(4,-12)=7(1,-3)+1(-3,9)$

		\item $\vec{0}$ como combinación lineal de $(1,0,-1)$, $(0,-1,-1)$ ~y~ $(1,1,1)$
		\answer Al plantear $(0,0,0)=a (1,0,-1) + b (0,-1,-1) + c(1,1,1)$ obtenemos el sistema $\SEL{a+c=0 \\ -b+c=0 \\ -a-b+c=0}$. \\Sustituyendo $a=-c$ y $b=c$ en la tercer ecuación obtenemos $c=0$, y luego $a=0$ y $b=0$. Por lo tanto, $\vec{0}=0(1,0,-1)+0(0,-1,-1)+0(1,1,1)$

		\item $\vec{v}=3\ivec+\jvec$ a partir de $\vec{w}=(4,-6)$ y $\vec{u}=\left(-1,\f{3}{2}\right)$
		\answer Planteamos $(3,1)=a(4,-6)+b\left(-1,\f{3}{2}\right)$ y se forma el sistema $\SEL{4a-b=3 \\ -6a+\f{3}{2}b=1}$. \\Reemplazando la primera ecuación en la segunda vemos que da un absurdo: el sistema no tiene solución. Por lo tanto, no es posible escribir $\vec{v}$ como combinación lineal de $\vec{w}$ y $\vec{u}$.

		\item $\vec{r}=\SEL{ \rho_{\vec{r}} = 8 \\ \theta_{\vec{r}} = 120\degs\\ r_z = 7 }$ como combinación lineal de $\{\ivec, \jvec, \kvec\}$
		\answer En coordenadas cartesianas, $\vec{r}=(8\cos{120\degs},8\sin{120\degs},7)=(-4,4\sqrt{3},7)$. Por lo tanto, $\vec{r}=-4\ivec+4\sqrt{3}\jvec+7\kvec$.
	\end{enumcols}


	\exercise Calcular los siguientes productos vectoriales e interpretarlo gráficamente.
	\begin{enumcols}[2]
		\item $(1,2,0) \times (0,1,3)$
		\answer $(6,-3,1)$

		\item $(3\ivec) \times (4\kvec)$
		\answer $(3\ivec) \times (4\kvec)=12 (\ivec \times \kvec)= -12 \jvec$

		\item $\left(\f{1}{4},1,-1\right) \times \left(1,5,-3\right)$
		\answer $\left(2,-\f{1}{4},\f{1}{4}\right)$

		\item $(1,1,2) \times (7,7,14)$
		\answer $(0,0,0)$. Por lo tanto, los vectores son paralelos.

		\item $(10~\vec{r}) \times \vect{F}$ ~con $\vec{r}=(1,2,0)$ ~y~ $\vect{F}=(3,-1,0)$
		\answer Como $\vec{r} \times \vect{F} = (0,0,-7)$, entonces $(10\vec{r}) \times \vect{F}=10 \left(\vec{r} \times \vect{F}\right) = 10 (0,0,-7) = (0,0,-70)$

		\item $\vec{a} \times \vec{b}$ ~con $\vec{a}=5\ivec$ ~y~ $\vec{b}=-2\jvec$
		\answer $(5\ivec)\times(-2\jvec)=-10(\ivec \times \jvec)=-10\kvec$

		\item $\vec{r} \times \vec{p}$ ~con $\vec{r}=5\jvec$ ~y~ $\vec{p}=\jvec+7\kvec$
		\answer $(5\jvec) \times (\jvec+7\kvec)= 5 (\jvec \times \jvec) + 35 (\jvec \times \kvec)= \vect{0}+ 35 \ivec=35 \ivec$

		\item $\vec{v} \times \vect{AB}$ ~con $\vec{v}=(-1,10,2)$, $A(4,1,3)$ ~y~ $B(4,-6,0)$
		\answer Calculamos $\vect{AB}=(0,-7,-3)$ y luego $\vec{v} \times \vect{AB}=(-16,-3,7)$

		\item $\vec{v} \times \vec{w}$ y $\vec{w} \times \vec{v}$~con $\vec{v}=(1,2,3)$ y $\vec{w}=(4,5,6)$
		\answer La primera operacion es $\vec{v} \times \vec{w}=(-3,6,-3)$, y la segunda es $\vec{w} \times \vec{v}=(3,-6,3)$. La segunda es el vector opuesto de la primera.

	\end{enumcols}


	\exercise Considerando los vectores $\vec{v}, \vec{u} , \vec{w}$, indicar si esposible realizar las siguientes operaciones en $\R^2$, $\R^3$, en ambos o en ninguno, y qué resultado se espera (número escalar o vector). 
	\begin{enumcols}[3]
		\item $2(\vec{v}+\ivec)$
		\answer Se puede realizar en $\R^2$ y en $\R^3$. El resultado es un vector.

		\item $\f{1}{|\vec{v}|} (\vec{v} \cdot \vec{u})$
		\answer Se puede realizar en $\R^2$ y en $\R^3$ siempre y cuando $\vec{v}$ no sea el vector nulo. El resultado es un número escalar.

		\item $\kvec+\vec{v} \cdot \ivec$
		\answer No se puede realizar, sería la suma de un vector y un número escalar.

		\item $\vec{v} \cdot (\vec{u} + \vec{w})$
		\answer Se puede realizar en $\R^2$ y en $\R^3$. El resultado es un número escalar.

		\item $\vec{w} \cdot (\vec{v} \times \vec{u})$
		\answer Se puede realizar en $\R^3$. El resultado es un número escalar.

		\item $(\vec{v} \cdot \vec{u}) ~\vec{w}$
		\answer Se puede realizar en $\R^2$ y en $\R^3$. El resultado es un vector.

		\item $\f{\vec{v} \cdot \vec{v}}{\vec{u} \cdot \vec{u}}$
		\answer Se puede realizar en $\R^2$ y en $\R^3$ siempre y cuando $\vec{u}$ no sea el vector nulo. El resultado es un número escalar.

		\item $\f{\vec{v} \cdot \vec{v}}{\vec{u}}$
		\answer No se puede realizar, sería la división de un número por un vector.

		\item $\f{\vec{v}^{~2}}{|\vec{v}|}$
		\answer No se puede realizar, no se define la potenciación en vectores, por lo tanto no existe el cuadrado de un vector.

		\item $\f{\vec{v} \cdot \vec{u}}{ |\vec{u}|^2} ~ \vec{u}$
		\answer Se puede realizar en $\R^2$ y en $\R^3$ siempre y cuando $\vec{u}$ no sea el vector nulo. El resultado es un vector.

		\item $(\vec{v} \times \vec{u}) \times \vec{w}$
		\answer Se puede realizar en $\R^3$. El resultado es un vector.

		\item $(\ivec \cdot \jvec) ~ (\kvec \times \vec{v})$
		\answer Se puede realizar en $\R^3$. El resultado es un vector.

		\item $\jvec+\vec{u} \times \kvec$
		\answer Se puede realizar en $\R^3$. El resultado es un vector.

	\end{enumcols}


	\exercise Considerando el concepto de proyección vectorial, escalar y componente ortogonal, calcular e interpretar gráficamente:
	\begin{enumcols}
		\item La proyección vectorial y escalar de $\vec{v}=(5,4)$ sobre $\vec{u}=(2,1)$
		\item La proyección vectorial y el componente ortogonal de $(-1,3)$ en la dirección de $(5,-1)$
		\item La proyección vectorial y el componente ortogonal de $\vec{r}=(-1,3)$ sobre $\vec{s}=(3,1)$
		\item La proyección vectorial y escalar de $\vec{v}=(1,5,1)$ en la dirección de $\vec{u}=(-1,2,3)$
		\item La proyección escalar y el módulo del componente ortogonal $\vect{F}=(7,-13)$ sobre $\ivec$
		\item La proyección vectorial de $\vec{a} \in \R^2: \SEL{|\vec{a}|=9 \\ \theta=120\degs}$ sobre $\vec{b} \in \R^2: \SEL{|\vec{b}|=5 \\ \theta=0\degs}$
		\item La proyección vectorial, escalar, el componente ortogonal y su módulo del vector $\vec{v}=(1,2,3)$ en la dirección del vector $\vect{AB}$, con $A(0,-1,4)$ y $B(2,2,3)$
		\item La proyección vectorial y escalar de $\vec{v}=(6,2)$ en la dirección de la recta $y=\f{1}{3}x+5$
		\item La proyección vectorial y su módulo del vector $\vect{AB}$ en la dirección de $\vect{AC}$ considerando $A(-1,3)$, $B(1,2)$ y $C(5,4)$.
	\end{enumcols}


	\exercise Considerando las propiedades geométricas del producto vectorial, determinar:
	\begin{enumcols}
		\item El área del paralelogramo de lados $\vec{u}=(3,-1,4)$ y $\vec{v}=(6,-2,8)$
		\item El área del triángulo de lados $\vec{a}=(2,3,0)$ y $\vec{b}=(-1,2,-2)$
		\item El área del el triángulo de vértices $A(2,6,-1)$, $B(1,1,1)$ y $C(4,6,2)$
		\item Si los vectores $\vec{r}=(1,2,3)$ y $\vec{s}=(3,6,9)$ son paralelos 
		\item Si los puntos $A(1,2,3)$, $B(2,3,4)$, $C(3,4,5)$ están alineados
	\end{enumcols}


	\exercise Considerando las propiedades geométricas del producto mixto, determinar:
	\begin{enumcols}
		\item El volumen del paralelepípedo  cuyos lados son los vectores $\vec{v}=(2,-6,2)$, $\vec{u}=(0,4,-2)$ y $\vec{w}=(2,2,-4)$
		\item El volumen del parallelepípedo cuyos lados son los vectores $\vec{a}=(3,1,2)$, $\vec{b}=(4,5,1)$ y $\vec{c}=(1,2,4)$
		\item Si los vectores $\vec{r}=(-1,-2,1)$, $\vec{s}=(3,0,-2)$ y $\vec{t}=(5,-4,0)$ son coplanares
		\item Si los puntos $A(0,0,0)$, $B(5,-2,1)$, $C(4,-1,1)$ y $D(1,-1,0)$ son coplanares
		\item Si alguno de los vectores $\vec{a}=(4,-8,1)$, $\vec{b}=(2,1,-2)$, $\vec{c}=(3,-4,12)$ es combinación lineal de los otros dos
	\end{enumcols}


	\exercise Resolver los siguientes problemas geométricos y analíticos:
	\begin{enumcols}
		\item Considerando el paralelepípedo cuyos lados son los vectores $\vec{u}=(1,2,0)$, $\vec{v}=(1,4,0)$ y $\vec{w}=(3,2,5)$, calcular el volumen, el área de la base y la altura. Interpretar geométricamente la situación.
		\item Demostrar que el triángulo de vértices $A(3,0,2)$, $B(4,3,0)$ y $C(8,1,-1)$ es rectángulo e indicar en qué vértice está el ángulo recto.
		\item Sean los vectores $\vec{a}=(6,-2,m)$ y $\vec{b}=\left(-9,3,-\f{3m}{2}\right)$, indagar si existe $m \in \R$ de modo que los vectores sean ortogonales, y si existen valores para que sean paralelos.
		\item Sean los vectores $\vec{u}=(1,1,k)$ y $\vec{v}=(2,3,1)$, determinar $k$ para que los vectores formen un paralelogramo de área $\sqrt{3}$. Graficar.
		\item Analizar si existen valores de $m \in \R$ tal que los vectores $\vec{v}=(1,2,3)$, $\vec{u}=(-1,4,m)$ y $\vec{w}=(1,3,2)$ sean coplanares.
		\item Sean los vectores $\vec{u}=(2,t)$, $\vec{v}=(3,5)$, investidar si existe $k \in \R$ tal que el ángulo entre $\vec{u}$ y $k\vec{v}$ sea de $45\degs$.
		\item Un vector forma ángulos iguales con los vectores $\ivec$, $\jvec$ y $\kvec$. Determinar las posibles componentes del vector.
		\item Averiguar si existen vectores de $\R^2$ de módulo 5 cuya proyección vectorial sobre $(2,1)$ es el vector $\left(\f{8}{5},\f{4}{5}\right)$. Interpretar gráficamente.
		\item Calcular los valores reales de $\alpha$ y $\beta$ para que los puntos $A(1,1,1)$, $B(\alpha,2,\beta)$ y $C(1,0,0)$ estén alineados.
		\item Los vectores $\vec{v}$ y $\vec{u}$ forman unángulo de $45\degs$. Si el módulo de $\vec{v}$ es $3$, calcular cuánto debe ser la medida de $\vec{u}$ para que $\vec{v}-\vec{u}$ resulte perpendicular a $\vec{v}$.
		\item Hallar $\vec{v} \cdot \vec{u}$ sabiendo que $\vec{v}=(1,-2)$, que $|\vec{u}|=4$ y que el ángulo entre ellos es $60\degs$.
		\item Obtener $(2\vec{u}-3\vec{v})\cdot (3\vec{u}+\vec{v})$ sabiendo que $\vec{u}$ es unitario, $|\vec{v}|=2$ y que $\vec{v} \cdot \vec{u} = 2$.
		\item Calcular el valor de $m$ para que el vector $(4,6,7)+m\ivec$ sea perpendicular a $(2,0,-1)$.
		\item Hallar los valores de $a,b \in\R$ para que el vector $(2,0,-1)+a\ivec$ sea paralelo a $(1,-1,3)+b\jvec$.
		\item Determinar los valores de $y,z \in\R$ para que el vector $(1,y,z)$ sea perpendicular a $(2,0,-1)$ y a $(1,-1,3)$.
		\item Encontrar el valor de $t$ para que los vectores $(2,0,-1)$ y $(1,t-1,3)$ tengan el mísmo módulo.
		\item Alicia y Beto salieron a caminar en un parque con brújulas. Alicia camina 2km en sentido noreste por un sendero recto que está a $60\degs$ del norte. Beto camina 3km al este y 1km al norte. ¿A qué distancia terminan? Beto quiere volver al sendero por el camino más corto (el perpendicular al sendero) ¿Qué dirección debe tomar? (responder indicando el ángulo respecto del norte).
		\item Demostrar que, si $\vec{v}$ es perpendicular a $\vec{u}$ y a $\vec{w}$ entonces $\vec{v}$ es perpendicular a $a\vec{u}+b\vec{w}$ para $a\neq0$ y $b\neq0$.
		\item Dado $\vec{v}\in \R^2$, demostrar que $|k\vec{v}|=|k|.|\vec{v}|$.
		\item Dados $\vec{v}, \vec{u} \in \R^2$, demostrar que $\f{\vec{v} \cdot \vec{u}}{|\vec{u}|}=\vec{v} \cdot \vers{u}$.

	\end{enumcols}

	\exercise Resolver los siguientes ejercicios de desafío:
	\begin{enumcols}
		\item Dados los vectores del plano $\vect{F}=\SEL{\left|\vect{F}\right|=10 \\ \theta_{\vect{F}}=120\degs}$ ~~y~~ $\vect{T}=\SEL{\left|\vect{F}\right|=16 \\ \theta_{\vect{T}}=-45\degs}$. \\ Calcular y graficar $\vect{F}+\vect{T}$, $\vect{F}-\vect{T}$ y el versor asociado a $\vect{F}$

		\item Dados los puntos $A(1,0,2)$ y $B(-1,-4,3)$, encontrar un vector perpendicular a $\vect{AB}$, uno paralelo y uno que forme un ángulo $30 \degs$.

		\item El vector $\vec{v}=(a^2,a-1,0) \in \R^3$ está definido por $a\in \R$. Por ejemplo, si $a=4$ entonces $\vec{v}=(16,3,0)$. Averiguar si existen valores de $a$ para los cuales $|\vec{v}|=1$. Por otra parte, para $a=2$ hallar un vector que sea simultáneamente perpendicular a $\vec{v}$ y a $\vec{u}=3\jvec-2\kvec$, y graficar la situación. Finalmente, para $a=3$ averiguar si se puede expresar el vector $\vec{v}$ como combinación lineal de los vectores $\vec{p}=(1,-6,0)$, $\vec{q}=(4,4,0)$ y $\vec{r}=(0,0,3)$

		\item Dados $\vec{u}=(2,-1,3)$ y $\vec{v}=(1,2,0)$. Hallar el ángulo entre $\vec{u}$ y $\vec{v}$, la proyección de $\vec{u}$ sobre $\vec{v}$ y su componente ortogonal. Graficar la situación.

		\item Averiguar si los puntos $A(1,-3,4)$, $B(5,1,4)$ y $C(-3,2,4)$ están alineados. En caso de que sí, encontrar un cuarto punto alineado. En caso de que no, encontrar un vector perpendicular a $\vect{AB}$ y a $\vect{AC}$.

		\item Hallar un vector simultáneamente perpendicular a $\vec{v}=(2,-3,0)$ y a $\vec{u}=(1,4,1)$ y graficar la situación. Describir qué relación tiene el vector hallado con el producto vectorial $\vec{u} \times \vec{v}$.

		\item Dados los vectores del plano $\vec{v}=(v_1,v_2)$ y $\vec{u}=(u_1,u_2)$ y la definición de producto escalar $\vec{v}\cdot \vec{u}=v_1 u_1+ v_2 u_2$, demostrar que $\vec{v} \cdot \vec{v} = |\vec{v}|^2$ y también que $\vec{v}\cdot(7\vec{u})=7(\vec{v}\cdot\vec{u})$

		\item Averiguar si el vector del plano $\vec{v}=-\ivec+3\jvec$ es paralelo, perpendicular o forma un ángulo con el vector $\vec{w}$ que tiene módulo 4 y ángulo $\theta=-\f{\pi}{3}$. Detallar la situación lo mejor posible.

		\item Dados los puntos $A(2,1)$ y $B(4,0)$. Calcular la proyección vectorial y el componente ortogonal del vector $\vec{u}=(2,2)$ sobre la dirección de $\vect{AB}$. Graficar la situación.

		\item Considerando los vectores $\vec{v}=(1,2,3)$ y $\vec{u}=(1,0,2)$. Calcular el producto vectorial $\vec{w}=\vec{v}\times \vec{u}$, comprobar que $\vec{w} \perp \vec{v}$ y graficar los tres vectores.
		
	\end{enumcols}

	% Finalmente ejercicios aplicador, recordar fuerza palicada sobre una pared en angulo, etc
\end{enumerate}

\end{document}