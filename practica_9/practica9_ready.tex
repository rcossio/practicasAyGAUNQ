\documentclass[a4paper]{article}
\usepackage[margin=1.5cm]{geometry}
\usepackage{multicol}
\usepackage{enumitem}
\usepackage{graphicx}
%Links
\usepackage[colorlinks = true,
            linkcolor = blue,
            urlcolor  = blue,
            citecolor = blue,
            anchorcolor = blue]{hyperref}
%Simbolos matemáticos
\usepackage{amsmath}
\usepackage{amssymb}
%Enumeracion
\usepackage{enumitem}
%Páginas sin numeración
\pagestyle{empty}
%Interlineado
\renewcommand{\baselinestretch}{1.5}
%Arreglar comillas
\usepackage [autostyle]{csquotes}
\MakeOuterQuote{"}
%Macros
\newcommand{\Item}{\item[\stepcounter{enumii}$\blacktriangleright$\textbf{(\alph{enumii})}]} %Negrita en algunos items
\newcommand{\answer}{\item[**]}
\newcommand{\exercise}{\item}
\newcommand{\SEL}[1]{\left\{\begin{matrix} #1 \end{matrix}\right.}
\newcommand{\df}[2]{\displaystyle\frac{#1}{#2}}
\newcommand{\conj}[1]{\overline{#1}}
\newcommand{\cis}[1]{\left[\cos\left({#1}\right)+i\sin\left({#1}\right)\right]}
\newcommand{\img}[2]{ \begin{minipage}[t]{\linewidth} \raisebox{-\height}{\includegraphics[width=#1]{#2}} \end{minipage} }
\begin{document}
\noindent \hrulefill 
\vspace{-7pt}
\begin{center} 
	\textbf{ Práctica 9: Superficies } \\
	Comisión: Rodrigo Cossio-Pérez y Gabriel Romero
\end{center}
\vspace{-10pt}
\hrulefill
\begin{enumerate}
	\exercise Clasificar las siguientes superficies cuadráticas mediante el análisis de trazas y graficarlas aproximadasmente.
	\begin{multicols}{2}
	\begin{enumerate} [label=(\alph*)]
		\item $x^2+y^2+z^2=9$
		\item $16x^2+4y^2+8z^2=32$
		\item $x^2+y^2-9z^2=9$
		\item $36x^2-9y^2-36z^2=36$
		\item $x^2+4y^2=-16z$
		\item $4x^2-9y^2=-36z$
		\item $x^2+y^2-z^2=0$
	\end{enumerate}
	\end{multicols}
	\exercise Identificar las siguientes superficies cilíndricas y graficarlas aproximadasmente.
	\begin{multicols}{2}
	\begin{enumerate} [label=(\alph*)]
		\item $y^2+z^2=4$
		\item $4x^2+25y^2=100$
		\item $z=4y^2$
		\item $x^2-y^2=1$
	\end{enumerate}
	\end{multicols}
	\exercise Las siguientes ecuaciones no representan superficies cuádricas ni cilíndricas. Hallar el lugar geométrico que representan. aproximadasmente.
	\begin{multicols}{2}
	\begin{enumerate} [label=(\alph*)]
		\item $(x-1)^2+(y-2)^2+(z-3)^2=0$
		\item $3x^2+5y^2+5z^2+75=0$
		\item $36y^2-25z^2=0$
		\item $3x^2+4y^2=0$
		\item $2z^2-8=0$
		\item $x^2=9$
	\end{enumerate}
	\end{multicols}
	\exercise Clasificar las siguientes superficies y hacer un gráfico aproximado de las mismas. Identificar ejes de simetría rotacional, planos de simetría, puntos de inversión y rango de posibles de $x$, $y$ y $z$.
	\begin{multicols}{2}
	\begin{enumerate} [label=(\alph*)]
		\item $4+x^2+y^2-4z+z^2=0$
		\item $-20+36x^2+9y^2-16z+4z^2=0$
		\item $2+2x^2+2y^2-z=0$
		\item $2-x+4y^2+4z^2=0$
		\item $x^2-8y+4y^2+16z=0$
		\item $2x-3y+z-3=0$
		\item $y-7=0$
		\item $-3+x^2+2y-y^2-4z=0$
		\item $-1-x+2y-y^2+z^2=0$
		\item $x^2+y^2-4=0$
		\item $x^2+z^2-4=0$
		\item $-2y^2+z=0$
		\item $1-16x+4x^2+2z+z^2=0$
		\item $-45-18x-9x^2+4y^2=0$
		\item $2x-3y+z=6$
		\item $-5+x^2+y^2-4z+z^2=0$
		\item $-20+36x^2-9y^2-16z+4z^2=0$
		\item $-20-36x^2-9y^2-16z+4z^2=0$
		\item $16+36x^2-9y^2-16z+4z^2=0$
	\end{enumerate}
	\end{multicols}
	\exercise Sea la ecuación $x^2+ky^2-4z^2=m$, identificar la superficie que se obtiene con
	\begin{multicols}{2}
	\begin{enumerate} [label=(\alph*)]
		\item $k=0$ y $m>0$
		\item $k<0$ y $m=0$
		\item $k>0$ y $m<0$
	\end{enumerate}
	\end{multicols}
	\exercise Dada la superficie $r.x^2+12y^2+s.z^2=144$, determinar los valores de $r,s \in \mathbb{R}$ para que la superficie sea:
	\begin{multicols}{2}
	\begin{enumerate} [label=(\alph*)]
		\item Una hipérbola de una hoja
		\item Una hiperbola de una hoja con sección circular
		\item Una hiperbola de dos hojas
		\item Una superficie cilíndrica recta de directriz hiperbólica
		\item Una elipsoide cuya intersección con el plano $x$-$z$ es una elipse de eje focal $z$, con semidiametro menor $3$ y semidiametro mayor $4$
		\item Una superficie cilindrica cuya intersección con el plano $y$-$z$ es una hipérbola con diámetros transverso y conjugado iguales 
	\end{enumerate}
	\end{multicols}
\end{enumerate}
\vspace{20pt} 
 \textbf{Respuestas}\begin{enumerate}\exercise---\exercise---\exercise---\exercise---\exercise---\exercise---\end{enumerate}\end{document}