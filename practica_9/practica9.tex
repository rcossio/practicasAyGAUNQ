
\documentclass[a4paper]{article}
\usepackage[margin=1.5cm]{geometry}

\usepackage{multicol}
\usepackage{enumitem}
\usepackage{graphicx}

%Links
\usepackage[colorlinks = true,
            linkcolor = blue,
            urlcolor  = blue,
            citecolor = blue,
            anchorcolor = blue]{hyperref}

%Simbolos matemáticos
\usepackage{amsmath}
\usepackage{amssymb}

%Enumeracion
\usepackage{enumitem}

%Páginas sin numeración
\pagestyle{empty}

%Interlineado
\renewcommand{\baselinestretch}{1.5}

%Arreglar comillas
\usepackage [autostyle]{csquotes}
\MakeOuterQuote{"}

%Macros
\newcommand{\Item}{\item[\stepcounter{enumii}$\blacktriangleright$\textbf{(\alph{enumii})}]} %Negrita en algunos items
\newcommand{\answer}{\item[**]}
\newcommand{\exercise}{\item}
\newcommand{\SEL}[1]{ \left\{\begin{matrix} #1 \end{matrix}\right. }
\newcommand{\df}[2]{\displaystyle\frac{#1}{#2}}
\newcommand{\conj}[1]{\overline{#1}}
\newcommand{\cis}[1]{\left[\cos\left({#1}\right)+i\sin\left({#1}\right)\right]}
\newcommand{\img}[2]{ \begin{minipage}[t]{\linewidth} \raisebox{-\height}{\includegraphics[width=#1]{#2}} \end{minipage} }

\begin{document}

\noindent \hrulefill 
\vspace{-7pt}
\begin{center} 
	\textbf{ Práctica 9: Superficies } \\
	Comisión: Rodrigo Cossio-Pérez y Gabriel Romero
\end{center}
\vspace{-10pt}
\hrulefill


\begin{enumerate}

	\exercise Clasificar las siguientes superficies cuadráticas mediante el análisis de trazas y graficarlas aproximadasmente.
	\begin{multicols}{2}
	\begin{enumerate} [label=(\alph*)]
		
		\item $x^2+y^2+z^2=9$
		\item $16x^2+4y^2+8z^2=32$
		\item $x^2+y^2-9z^2=9$
		\item $36x^2-9y^2-36z^2=36$
		\item $x^2+4y^2=-16z$
		\item $4x^2-9y^2=-36z$
		\item $x^2+y^2-z^2=0$

	\end{enumerate}
	\end{multicols}


	\exercise Identificar las siguientes superficies cilíndricas y graficarlas aproximadasmente.
	\begin{multicols}{2}
	\begin{enumerate} [label=(\alph*)]
		
		\item $y^2+z^2=4$
		\item $4x^2+25y^2=100$
		\item $z=4y^2$
		\item $x^2-y^2=1$

	\end{enumerate}
	\end{multicols}

	\exercise Las siguientes ecuaciones no representan superficies cuádricas ni cilíndricas. Hallar el lugar geométrico que representan. aproximadasmente.
	\begin{multicols}{2}
	\begin{enumerate} [label=(\alph*)]
		
		\item $(x-1)^2+(y-2)^2+(z-3)^2=0$
		\answer Punto $(1,2,3)$ 

		\item $3x^2+5y^2+5z^2+75=0$
		\answer Conjunto vacío $\emptyset$

		\item $36y^2-25z^2=0$
		\answer Dos planos: $6y-5z=0$ y $6y+5z=0$

		\item $3x^2+4y^2=0$
		\answer La recta $(x,y,z)=(0,0,0)+k(0,0,1)$. Recordar que $z$ es libre.

		\item $2z^2-8=0$
		\answer Dos planos: $z=2$ y $z=-2$

		\item $x^2=9$
		\answer Dos planos: $x=3$ y $x=-3$

	\end{enumerate}
	\end{multicols}

	\exercise Clasificar las siguientes superficies y hacer un gráfico aproximado de las mismas. Identificar si existen ejes de simetría rotacional, planos de simetría, puntos de inversión y rango de valores posibles de $x$, $y$ y $z$.
	\begin{multicols}{2}
	\begin{enumerate} [label=(\alph*)]
		
		\item $4+x^2+y^2-4z+z^2=0$
		\item $-20+36x^2+9y^2-16z+4z^2=0$
		\item $2+2x^2+2y^2-z=0$
		\item $2-x+4y^2+4z^2=0$
		\item $x^2-8y+4y^2+16z=0$
		\item $2x-3y+z-3=0$
		\item $y-7=0$
		\item $-3+x^2+2y-y^2-4z=0$
		\item $-1-x+2y-y^2+z^2=0$
		\item $x^2+y^2-4=0$
		\item $x^2+z^2-4=0$
		\item $-2y^2+z=0$
		\item $1-16x+4x^2+2z+z^2=0$
		\item $-45-18x-9x^2+4y^2=0$
		\item $2x-3y+z=6$
		\item $-5+x^2+y^2-4z+z^2=0$
		\item $-20+36x^2-9y^2-16z+4z^2=0$
		\item $-20-36x^2-9y^2-16z+4z^2=0$
		\item $16+36x^2-9y^2-16z+4z^2=0$

	\end{enumerate}
	\end{multicols}


	\exercise Sea la ecuación $x^2+ky^2-4z^2=m$, identificar la superficie que se obtiene con
	\begin{multicols}{2}
	\begin{enumerate} [label=(\alph*)]
		
		\item $k>0$ y $m>0$
		\answer Un hiperboloide de una hoja con centro en el origen y con eje principal $z$.

		\item $k=0$ y $m>0$
		\answer Un cilindro hiperbólico de generatrices paralelas al eje $y$ donde la hipérbola generatriz tiene eje focal $x$.

		\item $k<0$ y $m=0$
		\answer Un cono de vértice en el origen y eje principal $x$.

		\item $k>0$ y $m<0$
		\answer Un hiperboloide de dos hojas con centro en el origen y con eje principal $z$.

	\end{enumerate}
	\end{multicols}


	\exercise Dada la superficie $r.x^2+12y^2+s.z^2=144$, determinar los valores de $r,s \in \mathbb{R}$ para que la superficie sea:
	\begin{multicols}{2}
	\begin{enumerate} [label=(\alph*)]
		
		\item Un hiperboloide de una hoja
		\answer $r>0$ y $s<0$ para obtener un hiperboloide con eje principal $z$, o $r<0$ y $s>0$ para obtener un hiperboloide con eje principal $x$.

		\item Una hiperboloide de una hoja con sección circular
		\answer $r=12$ y $s<0$ para obtener un hiperboloide con eje principal $z$, o $r<0$ y $s=12$ para obtener un hiperboloide con eje principal $x$.

		\item Una hiperboloide de dos hojas
		\answer $r<0$ y $s<0$ para obtener un hiperboloide con eje principal $y$.

		\item Una superficie cilíndrica recta de directriz hiperbólica
		\answer $r=0$ y $s<0$ para obtener un cilindro hiperbólico con generatrices paralelas al eje $x$, o $r<0$ y $s=0$ para obtener un cilindro hiperbólico con generatrices paralelas al eje $z$.

		\item Una elipsoide cuya intersección con el plano $xz$ es una elipse de eje focal $z$, con semidiametro menor $3$ y semidiametro mayor $4$

		\item Una superficie cilindrica cuya intersección con el plano $yz$ es una hipérbola con diámetros transverso y conjugado iguales 
		\answer El plano $yz$ tiene ecuación $x=0$ por lo que la intersección es $12y^2+s.z^2=144$. Para que sea una hiperbola con diámetros transverso y conjugado iguales $s=12$. Para que sea una superficie cilíndrica $r=0$.

	\end{enumerate}
	\end{multicols}

	\exercise Parametrizar las siguientes superficies y realizar un esquema del significado de los parámetros.
	\begin{multicols}{2}
	\begin{enumerate} [label=(\alph*)]

		\item $(x-3)^2+y^2+(z-2)^2=49$
		\answer En coordenadas esféricas, utilizando la colatitud $u \in \left[0,\pi\right]$ y el ángulo azimutal $v \in \left[0,2\pi\right)$: \\ $\SEL{x=7\sin(u)\cos(v)+3 \\ y=7\sin(u)\sin(v) \\ z=7 cos(u)}$

		\item $\df{x^2}{4}+\df{y^2}{9}+\df{(z-2)^2}{1}=1$
		\answer Basado en la parametrización de una esfera utilizando $u \in \left[0,\pi\right]$ (colatitud) y $v \in \left[0,2\pi\right)$ (ángulo azimutal): \\ $\SEL{x=2\sin(u)\cos(v) \\ y=3\sin(u)\sin(v) \\ z= \cos(u)+2}$

		\item $\df{x^2}{9}+\df{y^2}{9}-\df{z^2}{4}=1$
		\answer Parametrizando con funciónes hiperbólicas utilizamos $u \in \mathbb{R}$ (ángulo hiperbólico) y $v \in \left[0,2\pi\right)$ (ángulo asimutal): \\ $\SEL{x=3\cosh(u)\cos(v) \\ y=3\cosh(u)\sin(v) \\ z=3\sinh(u)}$

		\item $-\df{x^2}{4}+\df{y^2}{1}-\df{z^2}{4}=1$
		\answer Parametrizando la mitad que corresponde a $+y$ con funciones hiperbólicas, utilizamos $u \in \mathbb{R}$ (ángulo hiperbólico) y $v \in \left[0,2\pi\right)$ (ángulo con eje $x$ en el plano $xz$): \\ $\SEL{x=2\sinh(u)\cos(v) \\ y=\cosh(u) \\ z=2\sinh(u)\sin(v)}$. \\ La mitad que corresponde a $-y$ es: \\ $\SEL{x=2\sinh(u)\cos(v) \\ y=-\cosh(u) \\ z=2\sinh(u)\sin(v)}$

		\item $4x^2+y^2-8z=0$
		\answer Utilizando $x=u \in \mathbb{R}$ e $y=v\in \mathbb{R}$ como parámetros: \\ $\SEL{x=u \\ y=v \\ z=\df{u^2}{2}+\df{v^2}{8}}$

		\item $4x^2-9y^2-36z=0$
		\answer Utilizando $x=u \in \mathbb{R}$ e $y=v\in \mathbb{R}$ como parámetros: \\ $\SEL{x=u \\ y=v \\ z=\df{u^2}{9}-\df{v^2}{4}}$

		\item $5x^2+5y^2-z^2=0$
		\answer En coordenadas cilíndricas, utilizamos $u \in \left[0,+\infty\right)$ (coordenada radial) y $v \in \left[0,2\pi\right)$ (ángulo azimutal): \\ $\SEL{x=u\cos(v) \\ y=u \sin(v) \\ z=5u}$

		\item $x^2+y^2=100$
		\answer En coordenadas cilíndricas, utilizamos $z=u \in \mathbb{R}$ y $v \in \left[0,2\pi\right)$ (ángulo azimutal): \\ $\SEL{x=10\cos(v) \\ y=10 \sin(v) \\ z=u}$

	\end{enumerate}
	\end{multicols}

\end{enumerate}

\end{document}