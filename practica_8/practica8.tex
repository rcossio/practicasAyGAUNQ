\documentclass{template_practica}

\begin{document}

\practiceheader{Práctica 8: Curvas cónicas}{Comisión: Rodrigo Cossio-Pérez y Gabriel Romero}

\begin{enumerate}

	\exercise Obtener analíticamente la ecuación general que debe cumplir un punto $P(x,y)$ sujeto a las condiciones geométricas indicadas. Describir el tipo de curva cónica.
	\begin{enumcols}
		
		\item La distancia de $P$ al origen de coordenadas es $5$
		\answer Planteando $d(P,O)=\sqrt{x^2+y^2}=5$ se obtiene la ecuación de la circunferencia centrada en el origen: $x^2+y^2-25=0$ 
		\item $d(P,C)=3$ con $C(1,5)$
		\answer Guía: Plantear la ecuación de distancia entre $P$ y $C$, y resolver el cuadrado del binomio. Analizar la condición geométrica para determinar de qué curva de trata y si su centro está en el origen o no.
		\item $d(P,F_1)+d(P,F_2)=10$ con $F_1(-3,0)$ y $F_2(3,0)$
		\answer Es una elipse de eje mayor horizontal: $16x^2 +25y^2 -400 =0$. \\ Resolución: \begin{align*} d(P,F_1)+d(P,F_2)&=10 \\ \sqrt{(x+3)^2+y^2}+\sqrt{(x-3)^2+y^2}&=10 \\ \sqrt{(x+3)^2+y^2}&=10 - \sqrt{(x-3)^2+y^2} \\ (x+3)^2+y^2&=100 - 20 \sqrt{(x-3)^2+y^2}+ (x-3)^2+y^2 \\ x^2+6x+9+y^2&=100 - 20 \sqrt{(x-3)^2+y^2}+ x^2-6x+9+y^2 \\ 20\sqrt{(x-3)^2+y^2} &=100-12x \\ 5\sqrt{(x-3)^2+y^2} &=25-3x \\ 25[(x-3)^2+y^2] &=625 -150x +9x^2 \\25x^2 -150x +225 +25y^2 &=625 -150x +9x^2 \\16x^2 +25y^2 -400 &=0\end{align*}
		\item La suma de las distancias de $P$ a $F_1$ y a $F_2$ es $8$ con $F_1(0,-2)$ y $F_2(0,2)$
		\answer Orientación: tener presente la distancia entre dos puntos, por ejemplo, $d(P,F_2)=\sqrt{x^2+(y-2)^2}$. Para eliminar una raíz de la ecuación se sugiere despejarla y elevar al cuadrado. Como la ecuación tiene dos raíces hay que realizar dicho procedimiento dos veces. Analizar la condición geométrica para determinar de qué curva de trata, la orientación de su eje principal, y si su centro/vértice está en el origen o no.
		\item $d(P,O)+d(P,A)=4$ donde $O$ es el origen de coordenadas y $A(2,0)$
		\item $d(P,F_1)+d(P,F_2)=6$ con $F_1(1,1)$ y $F_2(-1,-1)$ 
		\item La distancia de $P$ a $F(0,3)$ es igual a la de $P$ y la recta $r: y+3=0$
		\answer Es parábola con eje de simetría vertical: $x^2 -12y =0$. \\ Resolución: \begin{align*} d(P,F)&=d(P,r) \\ \sqrt{x^2+(y-3)^2}&=|y+3| \\ x^2+(y-3)^2&=(y+3)^2 \\ x^2+y^2-6y+9 &=y^2+6y+9 \\  x^2-12y&=0 \end{align*}
		\item $d(P,F)=d(P,L)$ con $F(2,0)$ y $L: x+2=0$
		\answer Orientación: tener presente la distancia entre puntos y la distancia de punto a recta. En este caso conviene trabajar directamente con el cuadrado de las distancias $d(P,F)^2=d(P,L)^2$ ya que evita considerar los casos del valor absoluto que surje de plantear la distancia de punto a recta. Analizar la condición geométrica para determinar de qué curva de trata, la orientación de su eje principal, y si su centro/vértice está en el origen o no.
		\item $d(P,F)=d(P,r)$ con $F(3,0)$ y $r: x-1=0$
		\item La distancia del $P$ a $F(1,-2)$ es igual a la distancia de $P$ al eje $x$
		\answer Recordatorio: El eje $x$ tiene ecuación $y=0$
		\item La distancia de $P$ al origen es igual a la de $P$ a $r: 3x+y+2=0$
		\item $|d(P,F_1)-d(P,F_2)|=2$ con $F_1(2,0)$ y $F_2(-2,0)$
		\answer Es una hipérbola con eje transversal horizontal: $3x^2-y^2-3=0$. Si $P$ está en el 1er o 4to cuadrante $d(P,F_1) > d(P,F_2)$ y podemos eliminar las barras de valor absoluto: \begin{align*} d(P,F_1)-d(P,F_2)&=2 \\ \sqrt{(x-2)^2+y^2}-\sqrt{(x+2)^2+y^2}&=2 \\ \sqrt{(x-2)^2+y^2}&=2+\sqrt{(x+2)^2+y^2} \\ (x-2)^2+y^2&=4+4\sqrt{(x+2)^2+y^2}+(x+2)^2+y^2 \\ x^2-4x+4+y^2&=4+4\sqrt{(x+2)^2+y^2}+x^2+4x+4+y^2 \\ -8x-4&=4\sqrt{(x+2)^2+y^2} \\ -2x-1&=\sqrt{(x+2)^2+y^2} \\ (2x+1)^2&=(x+2)^2+y^2 \\ 4x^2+4x+1&=x^2+4x+4+y^2 \\ 3x^2-y^2-3&=0 \end{align*} \\ Notar que si $P$ está en el 2do o 3er cuadrante $d(P,F_2) > d(P,F_1)$ y se llega a la misma ecuación: \begin{align*} d(P,F_2)-d(P,F_1)&=2 \\ \sqrt{(x+2)^2+y^2}- \sqrt{(x-2)^2+y^2}&=2 \\ \sqrt{(x+2)^2+y^2}&=2+\sqrt{(x-2)^2+y^2} \\ (x+2)^2+y^2&=4+4\sqrt{(x-2)^2+y^2}+(x-2)^2+y^2 \\ 8x-4&=4\sqrt{(x-2)^2+y^2} \\ (2x-1)^2&=(x-2)^2+y^2 \\ 4x^2-4x+1&=x^2-4x+4+y^2 \\ 3x^2-y^2-3&=0 \end{align*}
		\item El módulo de la diferencia entre distancias de $P$ a $F_1$ y a $F_2$ es $4$, con $F_1(0,3)$ y $F_2(0,-3)$
		\answer Sugerencia: Asumir que el punto $P$ está en el 2do o 3er cuadrante para obtener una expresión sin barras de valor absoluto
		\item $|d(P,F_1)-d(P,F_2)|=2$ con $F_1(1,1)$, $F_2(-1,-1)$ 

	\end{enumcols}

	\exercise Dar la ecuación canónica de las siguientes cónicas y graficarlas incluyendo elementos notables.
	\begin{enumcols}
		
		\item La circunferencia de centro $C(2,-3)$ y radio $r=4$. 
		\answer Orientación: Considerar la ecuación canónica de la circunferencia $(x-x_0)^2+(y-y_0)^2=r^2$

		\item La circunferencia de centro $C(1,-2)$ y que pasa por $P(6,5)$.
		\answer Guía: Conderar definición de "radio" de una circunferencia y la ecuación canónica de la circunferencia.

		\item La circunferencia cuyo diámetro es el segmento $PQ$ con $P(2,5)$ y $Q(6,5)$.
		\answer Guía: Considerar la definición de "diámetro" y la relación con el radio y con el centro de la circunferencia. Al obtener el radio y el centro de la circunferencia, utilizar la ecuación canónica de la circunferencia.

		\item La circunferencia que pasa por $P(0,3)$, $Q(2,4)$ y $R(4,3)$.
		\answer Orientación: Considerar la ecuación general de las cónicas con las condiciones para que sea circunferencia: $Ax^2+Ay^2+Dx+Ey+F=0$. Reemplazar los puntos para formar ecuaciones donde las incógnitas son $A$, $D$, $E$ y $F$. Resolver el sistema de ecuaciones para obtener los valores de las incógnitas. Para obtener la ecuación canónica se debera hacer completamiento de cuadrados.

		\item La circunferencia de centro $C(0,2)$ que es tangente a la recta $r: y=2x-1$.
		\answer Guía: Una recta es tangente a la circunferencia si la intersección con la misma es un único punto. Considerar el radio $r$ como un parámetro indeterminado y plantear la intersección de la recta y la circunferencia. Para que la intersección dé un único punto se puede pedir que el discriminante de la cuadrática sea $0$ y despejar $r$.

		\item La parábola de vértice $V(2,3)$ con directriz $x=-1$.
		\answer Orientación: Para plantear la ecuación canónica de la circunferencia, analizar si se trata de una parábola con eje de simetría vertical u horizontal, y hacia donde es su apertura. El valor del parámetro $p$ puede ser hallado a partir de la distancia entre el vértice y la directriz

		\item La parábola de foco $F(-1,3)$ y directriz $x=5$.
		\answer Guía: Analizar qué tipo de parábola es para plantear su ecuación canónica. El valor del parámetro $p$ puede ser hallado a partir de la distancia entre el foco y la directriz.

		\item La paŕabola de vértice $V(2,3)$ y foco $F(2,5)$.
		\answer Orientación: Analizar qué tipo de parábola es para plantear su ecuación canónica. El valor del parámetro $p$ puede ser hallado a partir de la distancia entre el vértice y el foco.

		\item La parábola con eje de simetría vertical que pasa por los puntos $P(0,1)$, $Q(6,4)$ y $R(2,0)$.
		\answer Guía: Conociendo que es una parábola con eje de simetría vertical se puede plantear la ecuación generál de cónicas con restricciones este tipo de parábolas $Ax^2+Dx+Ey+F=0$. Si se reemplazan los puntos se puede formar un sistema de ecuaciones para hallar $A$, $D$, $E$ y $F$. Para obtener la ecuación canónica se deberá hacer completamiento de cuadrados.

		\item La parábola con eje de simetría horizontal y concavidad hacia la derecha, foco $F(1,1)$ y que contiene al punto $P(7,9)$.
		\answer Orientación: Por la definición de la parábola la distancia del punto al foco es igual a la del punto a la directriz. Calculando $d(P,F)$ y conociendo el tipo de parábola de puede calcular la directriz. Luego se puede obtener el vértice y el parámetro $p$.

		\item La parábola con eje de simetría vertical y concavidad hacia abajo que pasa por los puntos $A(0,2)$ y $Q(4,0)$, y cuya distancia entre su foco y vértice es $1$. 
		\answer Guía: Con los datos puede obtenerse inmediatamente el parámetro $p$. Luego se puede plantear la ecuación canónica de la parábola aunque no conozcamos el vértice. Si se reemplazan los puntos de la parábola se forma un sistema de ecuaciones del cual se puede obtener el vértice. 

		\item La parábola con eje de simetría vertical, vértice $V(0,3)$ y que es tangente a la recta $r: y=2x+1$.
		\answer Orientación: Una recta es tangente a la parábola si la intersección con la misma es un único punto. Considerar el parámetro $p$ como un parámetro indeterminado y plantear la intersección de la recta y la parábola. Para que la intersección dé un único punto se puede pedir que el discriminante de la cuadrática sea $0$ y despejar $p$.

		\item La elipse de centro $C(2,3)$ y vértices $A(2,5)$ y $B(0,2)$.
		\answer Orientación: De los datos se pueden obtener los semiejes mayor y menor de la elipse, y obtener la ecuación canónica.

		\item La elipse de focos $F_1(2,5)$ y $F_2(2,1)$ y eje mayor $6$.
		\answer Guía: A partir de los focos se obtiene el centro de la elipse y el eje focal. A partir de los datos de pueden obtener los parametros $a$ y $b$ de la ecuación canónica de la elipse.

		\item La elipse de centro $C(-1,2)$, foco $F_1(-1,4)$, y vértice $V(-1,6)$.
		\answer Orientación: Con los datos pueden obtenerse los parámetros $a$ y $c$, y con ellos $b$ para plantear la ecuación canónica de la elipse. 

		\item La elipse con focos $F_1(5,2)$ y $F_2(1,2)$, y excentricidad $0.8$.
		\answer Guía: De los datos puede obtenerse el centro de la elipse y el semieje focal. Considerar la definicion de excentricidad para obtener otro parámetro de la elipse.

		\item La elipse de focos $F_1(-6,2)$ y $F_2(2,2)$, y que pasa por $P(2,8)$.
		\answer Orientación: De los datos puede obtenerse el centro, la distancia interfocal y el eje mayor. Estos datos se pueden procesar para obtener la ecuación canónica de la elipse. 

		\item La elipse de focos $F_1(2,5)$ y $F_2(2,1)$, y vértices $V_1(2,7)$ y $V_2(2,3)$.
		\answer Guía: De los datos puede obtenerse el centro, la distancia interfocal y el eje mayor. Estos datos se pueden procesar para obtener la ecuación canónica de la elipse.

		\item La elipse con vertices principales $V_1(3,0)$ y $V_2(13,0)$ y semidiámetro focal $3$.   
		\answer Orientación: De los datos puede obtener se el centro, y semieje mayor. Estos datos se pueden procesar para obtener la ecuación canónica de la elipse.

		\item La hipérbola de centro $C(2,3)$, foco $F_1(2,5)$, y vértice $V(2,7)$.
		\item La hipérbola de focos $F_1(2,5)$ y $F_2(2,1)$, y diámetro transversal $6$.
		\item La hipérbola de vértices $V_1(2,5)$ y $V_2(2,1)$, y diámetro conjugado $4$.
		\item La hipérbola de centro $C(2,3)$, focos $F_1(2,5)$ y $F_2(2,1)$, y excentricidad $1.5$.
		\item La hipérbola de centro $C(2,3)$, focos $F_1(2,5)$ y $F_2(2,1)$, y que pasa por $P(2,3)$.
		\item La hipérbola de focos $F_1(2,5)$ y $F_2(2,1)$, y vértices $V_1(2,7)$ y $V_2(2,3)$.
		\item La hipérbola cuyas asíntotas son $y=\pm 2x+1$ y vértice $V(1,1)$.

	\end{enumcols}


	\exercise Identificar las siguientes cónicas, dar su ecuación canónica, obtener sus elementos y graficarlas. Finalmente, dar una parametrización de las mismas.
	\begin{enumcols}[2]
		\item $4x^2+25y^2=100$
		\answer Orientación: Analizar los coeficientes cuadráticos para identificar la cónica. Operar algebraicamente cuando sea necesario para obtener la ecuación canónica.
		\item $y^2+2x=0$
		\item $9y^2-16x^2=144$
		\item $1-16x+4x^2+2y+y^2=0$
		\answer Orientación: Analizar los coeficientes cuadráticos para identificar la cónica. Una vez identificada completar cuadrados en $x$ y en $y$ y pasos algebraicos para orientar la expresión a la ecuación canónica correspondiente.
		\item $10-2x+x^2+10y+y^2=0$
		\item $3x^2-12x-y+5=0$
		\item $-25-32x+16x^2-10y-y^2=0$
		\item $y^2+2x=0$
		\item $2x-2y^2-12y+2=0$
		\item $x^2+y^2-8x=0$
		\item $-45-18x-9x^2+4y^2=0$
		\item $x^2+y^2=0$
		\answer Guía: Esta ecuación no concuerda con la de una circunferencia, ni parábola, ni elipse, ni hipérbola, por lo tanto es una cónica degenerada. Pensar que puntos $P(x,y)$ podrían cumplir la ecuación. Alternativamente pensar que se parece a una circunferencia pero donde el radio cobra valor $0$. 

		\item $9x^2+9y^2-24x+12y+11=0$
		\item $x^2+y^2-2x+4y+14=0$
		\answer Orientación: Al completar cuadrados encontramos la ecuación $(x-1)^2+(y+2)^2+9=0$. Pensar qué puntos de $\R^2$ cumplen esta ecuación. Recordar que todo número al cuadrado es mayor o igual que cero.
		\item $12x-y^2+10y-61=0$
		\item $9x^2-4y^2=36$
		\item $x^2-10x+8y+41=0$
		\item $4x^2+2y^2-8x-12y+14=0$
		\item $x^2-8x-6y-14=0$
		\item $x^2-4y^2-2x-3=0$
		\item $-4x^2+9x^2-8x+36y-40=0$

	\end{enumcols}

	\exercise Identificar y graficar las siguientes curvas relacionadas a las cónicas
	\begin{enumcols}[2]
		
		\item $\SEL{x=3\cos\theta \\ y=3\sin\theta}$ ~~~con $\theta\in[0,2\pi)$
		\answer Orientación: Es posible darle valores al parámetro en cuestión para obtener puntos de la curva. Alternativamente, identificar qué ecuación se puede armar con las expresiones de $x$ e $y$. En este caso se pueve verificar que $x^2+y^2=9$, basándonos en la identidad trigonométrica $\cos^2\theta+\sin^2\theta=1$.
		\item $\SEL{x=2\cos\theta \\ y=4\sin\theta}$ ~~~con $\theta\in[0,2\pi)$
		\item $\SEL{x=\cos\theta +2 \\ y=\sin\theta -2}$ ~~~con $\theta\in[0,\pi]$
		\item $\SEL{x=2\cos\theta -5 \\ y=3\sin\theta}$ ~~~con $\theta\in\left[\pi,\frac{3\pi}{2}\right)$
		\item $\SEL{x=\tau \\ y=2\tau^2-1}$ ~~~con $\tau\in[0,3)$
		\item $\SEL{x=\tau^2-2\tau+5 \\ y=\tau}$ ~~~con $\tau\in(-3,4)$
		\item $\SEL{x=\cosh\theta \\ y=\sinh\theta}$ ~~~con $\theta\in[0,+\infty)$
		\item $\SEL{x=2\cosh\theta \\ y=\sinh\theta -1}$ ~~~con $\theta\in\R$

	\end{enumcols}

	\exercise Hallar la analíticamente la intersección entre las siguientes curvas.
	\begin{enumcols}[2]
		
		\item $(x+4)^2+y^2=49$ y $x+y-5=0$
		\answer Sugerencia: Despejar una variable de la recta dada y sustituirla en la ecuación de la circunferencia.
		\item $3x^2-2y^2-1=0$ ~~y~~ $y-x^2+2=0$

		\item $\f{x^2}{9}+y^2-1=0$ ~~y~~ $2y-x^2-2=0$
		\answer Orientación: Despejar una variable de la parábola y sustituirla en la ecuación de la circunferencia.

		\item $\f{x^2}{9}-\f{y^2}{4}-1=0$ ~~y~~ $\f{x^2}{9}+\f{y^2}{4}-1=0$
		\answer Guía: Aplicar el método de eliminación convenientemente para resolver una de las variables.
		
		\item $y^2-x=0$ ~~y~~ $x^2+y^2-2x=0$

	\end{enumcols}

	\exercise Resolver las siguientes situaciones problemáticas relacionadas a cónicas.
	\begin{enumcols}
		
		\item Hallar el valor de $k\in\R$ para que la ecuación $x^2+y^2-8x+10y+k=0$ represente una circunferencia de radio $7$
		\item Encontrar una ecuación de las rectas paralelas al vector $\vec{v}=(2,-3)$ que son tangentes a la circunferencia $C: 2x^2+2y^2+4x+4y-22=0$
		\item La ecuación $x^2+y^2\leq 25$ describe el alcance de un radar de una estación de una guardia costera y la ecuación $(x,y)=(t,t-5)$ representa la trayectoria que sigue una embarcación. ¿En qué tramo de la trayectoria la embarcación será detectada por el radar? Realizar un esquema de la situación.
		\item Obtener los valores de $k\in\R$ para que la recta $y-kx+14=0$ sea tangente a la parábola $P: x^2-6x-6y-3=0$
		\item Dos postes de alumbrado ubicados en bordes opuestos de una carretera, distantes $8m$ entre sí y con $10m$ de altura cada uno, sostienen en sus extremos superiores un cable que forma un arco parabólico. Si el punto más bajo del cable está a $2m$ del suelo, hallar la ecuación de la parábola que describe el cable.
		\item Determinar las ecuaciones de las rectas tangentes a la elipse $3x^2+y^2+4x-2y-3=0$ que son perpendiculares a la recta $r: x+y-5=0$. Esquematizar la situación.
		\item Calcular la medida del ángulo agudo que forman las asíntotas de la hipérbola $H: 9x^2-y^2-36x-2y+44=0$
		\item Un barco envía una señal de auxilio en el momento en que se encuentra a $100km$ de la costa. Dos estaciones de guardacostas (A y B)ubicadas a $200km$ de distancia entre sí reciben la señal. A partir de la diferencia entre los tiempos de recepción de la señal se determina que la nave está a $160km$ más cerca de la estación B que de la A. Hallar la posición del barco en el momento de enviar la señal. Realizar un esquema de la situación.

	\end{enumcols}

\end{enumerate}

\end{document}