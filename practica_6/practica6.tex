\documentclass{template_practica}

\begin{document}

\practiceheader{Práctica 6: Determinantes}{Comisión: Rodrigo Cossio-Pérez y Gabriel Romero}

\begin{enumerate}

	\exercise Calcular los siguientes determinantes $\Delta$. Se puede utilizar cualquier método, tal como el de Sarrus y el de Laplace.
	\begin{enumcols}[3]
		
		\item $\Det{ 5 & -3 \\ 6 & 4}$
		\answer $\Delta=38$. Resolución por \href{https://youtu.be/bsUUVmeqsdY?t=104}{Matemáticas profe Alex}

		\item $\Det{ \f{1}{2} & -2 \\ \f{3}{4} & 4}$
		\answer $\Delta=\f{7}{2}$. Resolución por \href{https://youtu.be/bsUUVmeqsdY?t=237}{Matemáticas profe Alex}

		\item $\Det{ 2+3i & i \\ 2 & 1+i}$
		\answer $\Delta=(2+3i).(1+i)-(2)(i)=-1+3i$. 

		\item $\Det{ -2 & 4 & 5 \\ 6 & 7 & -3 \\ 3 & 0 & 2 }$
		\answer $\Delta=-217$. Resolución por \href{https://youtu.be/8OnOZvc5rFQ}{Matemáticas profe Alex}

		\item $\Det{ 2 & -1 & 3 \\ 4 & -2 & 0 \\ 1 & 3 & -4 }$
		\answer $\Delta=42$. Resolución por \href{https://youtu.be/0iXeaZwPkzo}{Álgebra para Todos}

		\item $\Det{ -1 & 2 & 0 & 3 \\ -13 & 4 & 0 & -8 \\ 5 & -1 & 1 & 2 \\ -9 & 5 & 0 & 0 }$
		\answer $\Delta=17$. Resolución por \href{https://youtu.be/a264lpe7I_A?t=383}{Susi Profe}

		\item $\Det{ 2 & -1 & 0 & 3 \\ 0 & 2 & 3 & 2 \\ 1 & -1 & 3 & 0 \\ -2 & 3 & 0 & -1 }$
		\answer $\Delta=-12$. Resolución por \href{https://youtu.be/qDyeSvFMbTg}{Álgebra para Todos}

		\item $\Det{ ~2 & ~4 & ~0 & ~2 & ~0 \\ ~3 & ~0 & ~5 & ~1 & ~0 \\ ~0 & ~6 & ~2 & ~0 & ~3 \\ ~4 & ~0 & ~2 & ~0 & ~4 \\ ~0 & ~6 & ~1 & ~0 & ~2 }$
		\answer $\Delta=-272$. Resolución por \href{https://youtu.be/fPIaDQ8TOs0}{Ktipio}
	
	\end{enumcols}



	\exercise Averiguar si la matriz A es inversible y, en caso de que lo sea, hallar la inversa.
	\begin{enumcols}[3]
		
		\item $A=\Mat{ 3 & 1 \\ 4 & 3 }$
		\answer $\det(A)=5$ por lo que $\exists A^{-1}=\Mat{ \f{3}{5} & \f{-1}{5} \\ \f{-4}{5} & \f{3}{5} }$.

		\item $A=\Mat{ 2 & 4 \\ 3 & 6 }$
		\answer $\det(A)=0$ por lo que $\nexists A^{-1}$.

		\item $A=\Mat{ 2 & 3 & 0 \\ -4 & 4 & 1 \\ 7 & 1 & -1}$
		\answer $\det(A)=-1$ por lo que $\exists A^{-1}=\Mat{ 5 & -3 & -3\\ -3 & 2 & 2 \\ 32 & -19 & -20 }$.

		\item $A=\Mat{ 2 & 4 & 1 \\ -5 & 4 & 5 \\ 3 & 1 & -1}$
		\answer $\det(A)=5$ por lo que $\exists A^{-1}=\Mat{ -\f{9}{5} & 1 & \f{16}{5}\\ 2 & -1 & -3 \\ -\f{17}{5} & 2 & \f{28}{5} }$.

		\item $A=\Mat{ 1 & 3 & 2 \\ -2 & 3 & 3 \\ -1 & 15 & 12 }$
		\answer $\det(A)=0$ por lo que $\nexists A^{-1}$.

		\item $D=\Mat{ 4 & 14 & -17 & 1 \\ 0 & 1 & 23 & 1 \\ 0 & 0 & 0 & 3 \\ 0 & 0 & 0 & 12 }$
		\answer $\det(D)=0$ por lo que $\nexists D^{-1}$.

	\end{enumcols}


	\exercise Hallar para qué valor/es (en $\C$) de la incógnita dada, el determinante toma el valor indicado.
	\begin{enumcols}[2]
		
		\item $\Det{ 3-\lambda & 0 \\ 8 & -1-\lambda }=0$
		\answer $\lambda=3$ o $\lambda=-1$. Resolución por \href{https://youtu.be/Rw531O86QVw?t=376}{Roberto Pintos}

		\item $\Det{ 4 & 2x-2 & 0 \\ 3 & 5 & 1 \\ 1 & 4 & x}=14$
		\answer $x=2$ o $x=\f{8}{3}$. Resolución por \href{https://youtu.be/mjS7OMrtNd8}{Mate Profesor Rosado}

		\item $\Det{ 2 & x & x+1 \\ x+2 & 2 & 8 \\ 2 & 1 & 3}=0$
		\answer $x=\displaystyle\f{9\pm\sqrt{33}}{4}$. Resolución por \href{https://youtu.be/nJRWcW-m7UU}{Profe Online}

		\item $\Det{ 1-\lambda & -1 & 0 \\ -1 & 2-\lambda & -1 \\ 0 & -1 & 1-\lambda}=0$
		\answer $\lambda=0$, $\lambda=1$ o $\lambda=2$. Resolución por \href{https://youtu.be/YCZd_BWyE0o}{Matemático Compulsivo}

	\end{enumcols}


	\exercise Hallar para qué valor/es (en $\C$) de la incógnita dada, la matriz $M$ es invertible.
	\begin{enumcols}[2]

		\item $\Mat{ k^3 & 2 \\ 8 & k }$
		\answer $M$ es invertible cuando $\det(M)\neq0$, es decir, para $k\in\C-\{2,-2,2i,-2i\}$.

		\item $\Mat{ 2 & \alpha & \alpha+1 \\ \alpha+2 & 2 & 8 \\ 2 & 1 & 3}$
		\answer $M$ es invertible cuando $\det(M)\neq0$, es decir, para $\alpha\neq\displaystyle\f{9\pm\sqrt{33}}{4}$. Resolución por \href{https://youtu.be/nJRWcW-m7UU}{Profe Online}.

		\item $\Mat{ 1 & 1-x & -1 \\ x+1 & 1 & -1 \\ 1 & 1 & x+2}$
		\answer $M$ es invertible cuando $\det(M)\neq0$, es decir, para $x\neq0$ y $x\neq-2$, o bien $x\in\C-\{0,-2\}$. Resolución por \href{https://youtu.be/KAt_M122xGw}{Profe Córdoba}.

		\item $\Mat{ 1 & 1 & \lambda \\ \lambda & 2 & -1 \\ 3 & 1 & 1}$
		\answer $M$ es invertible cuando $\det(M)\neq0$, es decir, para $\lambda\in\C-\{0,7\}$. Resolución por \href{https://youtu.be/Q1HjTmvsAGc}{Yo Soy Tu Profe}.

	\end{enumcols}


	\exercise Averiguar si los sistemas de ecuaciones tienen solución única y, en caso de que si, resolverlos por el método de Cramer.
	\begin{enumcols}[3]

		\item $\SEL{ 2x-3y=4 \\ -3y+y=-2 }$
		\answer $\Delta=-7$, $\Delta_x=-2$ y $\Delta_y=8$ por lo que la solución es $(x,y)=\left(\f{2}{7},-\f{8}{7}\right)$

		\item $\SEL{ 2x-3y+2z=1 \\ -3x+y+2z=\f{11}{2} \\ 2x+y-4z=0 }$
		\answer $\Delta=2$, $\Delta_x=-61$, $\Delta_y=-74$ y $\Delta_z=-49$ por lo que la solución es $(x,y,y)=\left(-\f{61}{2},-37,-\f{49}{2}\right)$

		\item $\SEL{ x-3y+2z=-3 \\ 5x+6y-z=13 \\ 4x-y+3z=8 }$
		\answer $\Delta=16$, $\Delta_x=-32$, $\Delta_y=80$ y $\Delta_z=112$ por lo que la solución es $(x,y,y)=(-2,5,7)$. Resolución por \href{https://youtu.be/lLPcHVAqY80}{Julio Profe}.

	\end{enumcols}


	\exercise Clasificar el sistema en compatible determinado, compatible indeterminado o incompatible a partir de los valores (en $\R$) de la incógnita.
	\begin{enumcols}[3]

		\item $\SEL{ k^3x+2y=1 \\ 8x+ky=2 }$
		\answer $\Delta=k^4-16=0$ cuando $k=\pm 2$. \\Con $k=2$ y $k=-2$ es SI. Con $k\in\R-\{2,-2\}$ es SCD.

		\item $\SEL{ a^2x+y=1 \\ x+y=a }$
		\answer $\Delta=a^2-1=0$ cuando $a=\pm 1$. \\Con $a=1$ es SCI, con $a=-1$ es SI y con $a\in\R-\{1,-1\}$ es SCD.

		\item $\SEL{ x+y+\lambda z=-1 \\ \lambda x+2y-z=3 \\ 3x+y+z=5 }$
		\answer $\Delta=\lambda^2-7\lambda=0$ cuando $\lambda=0$ o $\lambda=7$. \\Con $\lambda=0$ y $\lambda=7$ es SI y con $\lambda\in\R-\{0,-7\}$ es SCD.

		\item $\SEL{ x+my+3z=2 \\ x+y-z=1 \\ 2x+3y+mz=3 }$
		\answer Es SCD si $m\in \R-\{2,-3\}$. \\ Es SCI si $m=2$. Es SI si $m=-3$. Resolución por \href{https://youtu.be/jKLkTVmpmSk}{Cibermatex}.

		\item $\SEL{ \alpha x-y+2z=0 \\ -3x+4y+\alpha z=0 \\ 2x+y-3z=0 }$
		\answer $\Delta=-\alpha^2-14\alpha-13=0$ cuando $\alpha=-1$ o $\alpha=-13$. \\Con $\alpha=-1$ y $\alpha=-13$ es SCI (como es homogéneo no puede ser SI) y con $\alpha\in\R-\{-1,-13\}$ es SCD.

	\end{enumcols}


	\exercise Utilizar el determinante para calcular los productos vectoriales indicados.
	\begin{enumcols}
		
		\item $\vec{u} \times \vec{v}$ con $\vec{u}=(2,-1,1)$ y $\vec{v}=(-3,1,1)$.
		\answer $\vec{u} \times \vec{v}=\Det{ \hat{\imath} & \hat{\jmath} & \hat{k} \\ 2 & -1 & 1 \\ -3 & 1 & 1}=(-2,-5,-1)$. Resolución por \href{https://youtu.be/P0aD2zSXuC8}{lasmatematicas.es}

		\item $\vec{a} \times \vec{b}$ con $\vec{a}=3\hat{\imath}+5\hat{\jmath}-2\hat{k}$ y $\vec{b}=2\hat{\imath}-4\hat{\jmath}+3\hat{k}$.
		\answer $\vec{a} \times \vec{b}=\Det{ \hat{\imath} & \hat{\jmath} & \hat{k} \\ 3 & 5 & -2 \\ 2 & -4 & 3}=7\hat{\imath}-13\hat{\jmath}-22\hat{k}$. Resolución por \href{https://youtu.be/-JODKVdQ9H4}{Matemáticas Edgar Navia}

		\item $\vec{m} \times \vec{n}$ con $\vec{m}=-3\hat{\imath}-2\hat{\jmath}+5\hat{k}$ y $\vec{n}=6\hat{\imath}-10\hat{\jmath}-\hat{k}$.
		\answer $\vec{m} \times \vec{n}=\Det{ \hat{\imath} & \hat{\jmath} & \hat{k} \\ -3 & -2 & 5 \\ 6 & -10 & -1}=52\hat{\imath}+27\hat{\jmath}+42\hat{k}$. Resolución por \href{https://youtu.be/fmAhi1N-uL8}{JulioProfe}

		\item $(1,2,3) \times (-1,3,0)$.
		\answer $(1,2,3) \times (-1,3,0)=\Det{ \hat{\imath} & \hat{\jmath} & \hat{k} \\ 1 & 2 & 3 \\ -1 & 3 & 0}=(-9,-3,5)$. Resolución por \href{https://youtu.be/_5MyVA6znPQ}{Seletube}

	\end{enumcols}


	\exercise Demostrar las siguientes propiedades sobre matrices de transformación en el plano.
	\begin{enumcols}

		\item El determinante de la matriz de reflexión sobre el eje $x$ o $y$ vale $-1$.
		\answer $\det(M^{Ref}_x)=\Det{ -1 & 0 \\ 0 & 1 }=-1$. \\ $\det(M^{Ref}_y)=\Det{ 1 & 0 \\ 0 & -1 }=-1$.  

		\item El determinante de la matriz de rotación con ángulo $\alpha$ vale $1$.
		\answer $\det(M^{Rot}_{\alpha})=\Det{ \cos(\alpha) & -\sin(\alpha) \\ \sin(\alpha) & \cos(\alpha) }=\cos^2(\alpha)+\sin^2(\alpha)=1$.

		\item El determinante de la matriz de cizallamiento de factor $k$ en el eje $x$ o $y$ vale $1$
		\answer $\det(M^{Ciz}_x)=\Det{ 1 & k \\ 0 & 1 }=1$.  \\ $\det(M^{Ciz}_y)=\Det{ 1 & 0 \\ k & 1 }=1$.

		\item El determinante de la matriz de compresión/expansión de factor $k$ en el eje $x$ o $y$ vale $k$.
		\answer $\det(M^{C/E}_x)=\Det{ k & 1 \\ 0 & 1 }=k$.  \\ $\det(M^{C/E}_y)=\Det{ 1 & 0 \\ 0 & k }=k$.

	\end{enumcols}


	\exercise Dadas $A=\Mat{ 1 & 0 & 0 \\ -1 & 5 & 0 \\ -1 & -3 & 1 }$ y $B=\Mat{ 2 & 1 & -1 \\ 0 & 1 & 8 \\ 0 & 0 & -1 }$, calcular $\det(A)$, $\det(B)$ y los siguientes determinantes relacionados aprovechando el uso de propiedades.
	\begin{enumcols}
		
		\item $\det(A^{-1})$
		\answer $\det(A^{-1})=\displaystyle{}\f{1}{\det(A)}}=\f{1}{5}$.

		\item $\det\left((A.B)^T\right)$ 
		\answer $\det[(A.B)^T]=\det(A.B)=\det(A).\det(B)=5.(-2)=-10$.

		\item $\det(A+B)$
		\answer $\det(A+B)=\Det{ 3 & 1 & -1 \\ -1 & 6 & 8 \\ -1 & -3 & 0 }=55$ (no hay propiedades utiles para aplicar).  

		\item $\det(2.A^4)$ 
		\answer $\det(2.A^4)=2^3.\det(A^4)=8.\det(A)^4=8.5^4=5000$.

		\item $\det(-k.B^n)$ con $k\in\R$ y $n\in\mathbb{N}$. 
		\answer $\det(-k.B^n)=(-k)^3.\det(B^n)=-k^3.\det(B)^n=-k^3.(-2)^n$.

		\item $\det\left((A^T.B^{-1})^2\right)$
		\answer $\det\left((A^T.B^{-1})^2\right)=\det(A^T.B^{-1})^2=\left(\det(A^T).\det(B^{-1})\right)^2=\left(\det(A).\displaystyle{\f{1}{\det(B)}}\right)^2=\left(\f{5}{-2}\right)^2=\f{25}{4}$.

	\end{enumcols}


	\exercise Resolver los siguientes ejercicios integradores
	\begin{enumcols}
		
		\item Dada la matriz simétrica $A=\Det{ 1 & a+b & 0 \\ 2 & 5 & a \\ b & c & 3}$ calcular $\det(A)$.
		\answer Por la simetría $a=2$, $b=0$ y $c=2$; y $\det(A)=-1$. Resolución por \href{https://youtu.be/FhKTUHGGyhk}{Ing. E Darwin}

		\item Dada la matriz $B=\Det{ x & 3 & 1 \\ x+1 & 4 & 2 \\ x & 2-x^2 & 1}$ tal que $\det(2B)=160$, calcular $x\in\C$.
		\answer Como $\det(2B)=160$, entonces $\det(B)=20$. Obteniendo el determinante se obtiene $x^3-x^2+x-21=0$, con soluciones $x=3$, $x=-1+\sqrt{6}i$ y $x=-1-\sqrt{6}i$. Resolución por \href{https://youtu.be/ysZg6eVeoSY}{Mates con Andrés}

	\end{enumcols}

	% Falta algo con propiedades de determinantes (E7 aljinovich y E.11)
	% Falta algo con volumen de paralelepipedo, matrices semejantes y area del paralelogramo
	
\end{enumerate}

\end{document}