
\documentclass[a4paper]{article}
\usepackage[margin=1.5cm]{geometry}

\usepackage{multicol}
\usepackage{enumitem}
\usepackage{graphicx}

%Links
\usepackage[colorlinks = true,
            linkcolor = blue,
            urlcolor  = blue,
            citecolor = blue,
            anchorcolor = blue]{hyperref}

%Simbolos matemáticos
\usepackage{amsmath}
\usepackage{amssymb}

%Enumeracion
\usepackage{enumitem}

%Páginas sin numeración
\pagestyle{empty}

%Interlineado
\renewcommand{\baselinestretch}{1.5}

%Arreglar comillas
\usepackage [autostyle]{csquotes}
\MakeOuterQuote{"}

%Macros
\newcommand{\Item}{\item[\stepcounter{enumii}$\blacktriangleright$\textbf{(\alph{enumii})}]} %Negrita en algunos items
\newcommand{\answer}{\item[**]}
\newcommand{\exercise}{\item}
\newcommand{\SEL}[1]{\left\{\begin{matrix} #1 \end{matrix}\right.}
\newcommand{\df}[2]{\displaystyle\frac{#1}{#2}}
\newcommand{\conj}[1]{\overline{#1}}
\newcommand{\cis}[1]{\left[\cos\left({#1}\right)+i\sin\left({#1}\right)\right]}

\begin{document}

\noindent \hrulefill 
\vspace{-7pt}
\begin{center} 
	\textbf{ Práctica 2: Polinomios } \\
	Comisión: Rodrigo Cossio-Pérez y Gabriel Romero
\end{center}
\vspace{-10pt}
\hrulefill


\begin{enumerate}

	\exercise Realizar las operaciones polinomiales
	\begin{multicols}{2}
	\begin{enumerate} [label=(\alph*)]
		
		\item $2P(x)+3Q(x)$ \\con $P(x)=2x^2+3$ y $Q(x)=x^3-x-1$
		\item $4P(x).Q(x)$ \\con $P(x)=2x^2+3$ y $Q(x)=x^3-x-1$
		\item $P(x)^2$ \\con $P(x)=2x^2+3$
		\item $P(-2)+Q\left(\df{1}{2}\right)$ \\con $P(x)=2x^2+3$ y $Q(x)=x^3-x-1$

		\item $\df{P(x)}{Q(x)}$ \\con $P(x)=x^4+4$ y $Q(x)=x^2+2x+2$
		\answer \href{https://youtu.be/DWDi7BKAKbc}{Resolución}

		\item $\df{x^3+2x^2-x+10}{x+2}$
		\answer \href{https://youtu.be/bfCWsvZfFq0}{Resolución}

		\item $\df{P(x)}{Q(x)}$ \\con $P(x)=2x^4-2x^3-13x^2+8x+15$ y $Q(x)=x-3$
		\answer \href{https://youtu.be/W3HcTD4IC94}{Resolución}

		\item $\df{8x^4-4}{-2x^2+x-3}$
		\answer \href{https://youtu.be/0Dw3MAwrA34}{Resolución}

	\end{enumerate}
	\end{multicols}

	\exercise Calcular el los valores reales ($\alpha$, $\beta$, $m$, etc.) para que se cumplan las condiciones

	\begin{multicols}{2}
	\begin{enumerate} [label=(\alph*)]
		
		\item $\alpha P(x)^2 = Q(x) +\beta$ \\con $P(x)=x-1$ y $Q(x)=2x^2-4x$
		\item $\alpha P(x)-\beta Q(x) = -2x^2 +7x-3$ \\con $P(x)=x-1$ y $Q(x)=2x^2-4x$
		\item $gr\left(\alpha P(x)^2 - Q(x)\right)=gr\left(P(x)\right)$ \\con $P(x)=x-1$ y $Q(x)=2x^2-4x$
		\item $Q(\alpha)=P(4)$ \\con $P(x)=x-1$ y $Q(x)=2x^2-4x$

		\item $3x^2+mx+4$ tiene a 1 como raíz
		\answer \href{https://youtu.be/09D5Z3dcaXc}{Resolución}

		\item $x^3+ax^2+bx+5$ es divisible por $Q(x)=x^2+x+1$
		\answer \href{https://youtu.be/jE5a43IQ91E}{Resolución}

	\end{enumerate}
	\end{multicols}

	\exercise Factorizar los siguientes polinomios considerando raíces reales y luego complejas.

	\begin{multicols}{2}
	\begin{enumerate} [label=(\alph*)]
		
		\item $x^2+\sqrt{3}x-1$
		\item $8x^3-28x^2+14x+15$

		\item $x^5-1$
		\answer \href{https://youtu.be/ZMBXAdxOleM}{Resolución}

		\item $x^5+3x^4+3x^3+3x^2+2x$
		
		\item $x^4-\df{9}{2}x^3+21x-10$
		\answer \href{https://youtu.be/1V06bnuaadA}{Resolución}

		\item $x^7+18x^6+81x^5$
		\answer \href{https://youtu.be/Z1KatpJM2eU}{Resolución}

		\item $x^6-x^4-20x^2$
		
		\item $x^7-x$
		\answer \href{https://youtu.be/EQIEmdkGOZE}{Resolución}

		\item $x^4+2x^2-2$

		\item $x^6+3x^3-4$
		\answer \href{https://youtu.be/BCo0pxE288w}{Resolución}

	\end{enumerate}
	\end{multicols}

	\exercise Proponer un polinomio $P(x)$ que cumpla con las siguientes condiciones.

	\begin{multicols}{2}
	\begin{enumerate} [label=(\alph*)]
		
		\item Es de grado 4, 2 es raíz simple, -1 es raíz triple y $P(1)=24$
		\answer \href{https://youtu.be/_XVYatmUKBg}{Resolución}

		\item $P(x)$ es mónico y de grado 2. Además, $P(1)=3$ y $P(-4)=2$.
		\answer \href{https://youtu.be/LDpq_f-baPc}{Resolución}

		\item $P(x)$ es de grado 5, tiene coeficientes reales, $P(i)=0$ y $P(1+i)=0$

		\item $gr\left(P(x)\right)=6$, $P(x)$ tiene coeficientes reales y algunas de sus raíces son $i$ y $1+i$
		\answer \href{https://youtu.be/LDpq_f-baPc}{Resolución}


	\end{enumerate}
	\end{multicols}

	\exercise Resolver los siguientes ejercicios de desafío
	\begin{enumerate} [label=(\alph*)]
		
		\item Hallar las raíces del polinomio $E(x)=\df{x^4}{2}+x^3-\df{x}{2}-1$ y graficarlas. Obtener el cociente y el resto de dividir a $E(x)$ por $D(x)=(x+2)(x-3)$.

		\item Dados los polinomios $K(x)=ax+b$, $L(x)=cx^2+x$ y $M(x)=x.K(x)-L(x)$, hallas los coeficientes $a,b,c \in \mathbb{R}$ que hacen que el grado $M(x)$ sea 1, $L(-2)=10$ y que $\df{1}{2}$ sea raíz de $K(x)$.

		\item Proponer un polinomio $H(x)$ de grado 3 con coeficientes reales que tenga al número $2-i$ como raíz, que cumpla con $H(0)=10$ y que no tenga raíces reales positivas. Luego, obtener el cociente y resto de dividir $H(x)$ por $L(x)=x^2+1$

		\item Hallar todas las raíces del polinomio $P(x)=x^6-3x^3-4$ y graficarlas en el plano complejo

		\item Encontrar y graficar todas las raíces complejas del polinomio $P(x)=2x^3+7x^2+6x-5$ e indicar cuáles de ellas se encuentran en la región $R=\{ z \in \mathbb{C} ~/~ |z|<1 \}$ del plano complejo

		\item Dado el polinomio $P(x)=x^3-ix^2+4x-4i$: indicar porqué no se puede aplicar el método de Gauss para encontrar ls raíces de $P(x)$; indicar cuántas raíces tiene $P(x)$ (sin calcularlas); y, finalmente, demostrar que $i$ es raíz de $P(x)$ dividiéndolo por $(x-i)$ mediante Ruffini o división tradicional

		\item Encontrar los valores de $\alpha, \beta \in \mathbb{R}$ que permitan expresar el polinomio $P(x)=x-7$ como $P(x)=\alpha Q(x)+\beta R(x)$ considerando $Q(x)=2x$ y $R(x)=x$

	\end{enumerate}


\end{enumerate}

\end{document}